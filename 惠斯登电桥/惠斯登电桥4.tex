\documentclass[]{../template/Report}%方括号内写yuxi即生成预习报告
\settemplatedir{../template/}%设置模板路径

\exname{惠斯登电桥} %实验名称
\extable{} %实验桌号
\instructor{} %指导教师
\class{} %班级
\name{} %姓名
\stuid{} %学号

\nyear{} %年
\nmonth{} %月
\nday{} %日
\nweekday{} %星期几,e.g. \nweekday{三}
\daypart{}%上午/下午

\redate{} %如有实验补做,补做日期
\resitu{} %情况说明:
\begin{document}
\maketitle

\section{预习报告(10分)}
(注:将已经写好的“物理实验预习报告”内容拷贝过来)

\subsection{实验综述(5分)}
(自述实验现象、实验原理和实验方法,包括必要的光路图、电路图、公式等。不超过500字。)

如图所示,$R_1$、$R_2$、$R_s$ 和 $R_x$ 共同组成一个电桥电路,$R_x$ 为待测电阻,$R_s$ 为标准电阻,$R_1$ 和 $R_2$ 为比率臂电阻,$E$ 为直流电源,$G$ 为检流计。当电桥平衡时,检流计指针指零,此时有
\[
    R_x = \frac{R_1}{R_2} R_s
\]
\begin{figure}[H]
    \centering
    \includegraphics[width=0.6\textwidth]{微信图片_20251028221242_36_325.jpg}
    \caption{惠斯登电桥实验电路图}
    \label{fig:电桥}
\end{figure}
\subsubsection{实验原理}

当电桥平衡时,检流计电流\(I_G = 0\),即B、D两点电位相等(\(V_B = V_D\))。支路电流关系:\(I_1 = I_x\),\(I_2 = I_S\)。电压降关系:\(I_1 R_1 = I_2 R_2\),\(I_x R_x = I_S R_S\)。联立可得平衡条件:\(\frac{R_1}{R_2} = \frac{R_x}{R_S} \quad \text{或} \quad R_x = R_S \cdot \frac{R_1}{R_2}\)

\subsubsection{实验方法}

\begin{enumerate}
    \item 器材准备惠斯登电桥实验装置;待测电阻\(R_x\)、标准可变电阻\(R_S\)、已知比例的电阻\(R_1\)、\(R_2\);直流电源、开关、导线等。
    \item 实验步骤电路连接:按原理图连接四个电阻臂、检流计和电源。初步调节:选择合适的\(R_1/R_2\)比例,预估\(R_S\)的大致范围。
    \item 平衡调节:闭合电源开关,缓慢调节\(R_S\)的阻值,观察检流计指针偏转。若指针偏转方向变化,逐步缩小\(R_S\)的调节范围,直到检流计指针精确指零。
    \item 数据记录与计算:记录平衡时的\(R_1\)、\(R_2\)、\(R_S\),根据平衡条件计算\(R_x\)。多次测量:改变\(R_1/R_2\)的比例,取平均值以提高精度。
    \item 精度优化选择\(R_1/R_2\)为整数比例,简化计算并减小误差。确保检流计灵敏度足够,可通过串联保护电阻防止初始大电流损坏检流计,平衡前接入,平衡时移除。
\end{enumerate}

\subsubsection{实验现象}

\begin{enumerate}
    \item 当\(R_S\)未调节至平衡值时,检流计指针明显偏转
    \item 当\(R_S\)满足平衡条件时,检流计指针精确指零,此时无论怎样微调\(R_S\),指针均不再偏转。
    \item 调节\(R_S\)时,指针偏转角度逐渐减小,最终归零,直观体现了电桥从 “不平衡” 到 “平衡” 的过渡。
\end{enumerate}

\subsection{实验重点(3分)}
(简述本实验的学习重点,不超过100字。)
\begin{enumerate}
    \item 掌握惠斯登电桥工作原理及其特点,学会自组电桥测量未知电阻。
    \item 学习如何对测量结果进行误差分析。
\end{enumerate}
\subsection{实验难点(2分)}
(简述本实验的实现难点,不超过100字。)
\begin{enumerate}
    \item  检流计上的“电计”与“短路”按钮都具有锁定功能,测量时要确保“短路”按钮未锁定,否则检流计不会有偏转。
    \item 实验结束,关闭检流计和盒式惠斯登电桥。
\end{enumerate}

\begin{fullreportonly}
\section{原始数据(20分)}
(将有老师签名的“自备数据记录草稿纸”的扫描或手机拍摄图粘贴在下方,完整保留姓名,学号,教师签字和日期。)
\begin{figure}[H]
    \centering
    \includegraphics[width=0.8\textwidth]{微信图片_20251030170434_109_118.jpg}
    \caption{原始数据记录}
    \label{fig:原始数据}
\end{figure}
\section{结果与分析}
\subsection{数据处理与结果}
\subsubsection{交换法测量未知电阻}
实验一中,设置$R_1=R_2 = 8000 \si{\ohm}$。
测得$R_s = \SI{222.8}{\ohm},R_s = \SI{222.7}{\ohm}$。
可知此时倍率为$\frac{R_1}{R_2} = 1 $,待测电阻阻值为
\[\overline{R_x} = \sqrt{R_s R'_s} = \SI{222.7}{\ohm}\]

再测量电桥灵敏度,在$R_s$为$\SI{222.7}{\ohm}$时,调整$\Delta R_s = \SI{0.6}{\ohm}$,此时指针偏转了$\Delta d = 12$小格,在$R_s$为$\SI{222.8}{\ohm}$时,调整$\Delta R_s = \SI{0.6}{\ohm}$,此时指针偏转了$\Delta d = 11.5$小格,则电桥灵敏度为
\[S = \frac{\Delta d}{\frac{\Delta R_s}{R_s}} = \frac{12}{\frac{0.6}{222.7}} = 4454\]

$R_x$ 相对不确定度
\[E = \frac{\Delta R_x}{\overline{R_x}} = \sqrt{\left(0.001+ \frac{0.002m}{R_s}\right)^2 + \left(\frac{0.2}{S}\right)^2} = 0.10\% \]

则 $\Delta R_x = E \cdot \overline{R_x} = \SI{0.2}{\ohm}$。$R_x = (\overline{R_x} \pm \Delta R_x)\,\si{\ohm} = (222.7 \pm 0.2)\,\si{\ohm}$。
\subsubsection{用QJ\_23型盒式惠斯登电桥测电阻离散度}
实验中选择倍率为$0.1$,利用直接法就能算出各待测电阻阻值如\cref{yiduidianzu}所示。
\begin{table}[H]
    \caption{测得各电阻数据}
    \centering
    \begin{tabular}{C{0.1\textwidth}C{0.08\textwidth}C{0.08\textwidth}C{0.08\textwidth}C{0.08\textwidth}C{0.08\textwidth}C{0.08\textwidth}C{0.08\textwidth}C{0.08\textwidth}}
        \toprule
        待测电阻 & $R_1$ & $R_2$ & $R_3$ & $R_4$ & $R_5$ & $R_6$ & $R_7$ & $R_8$ \\
        \midrule
        电阻/$\si{\ohm}$ & 678.5 & 675.5 & 680.9 & 684.7 & 679.0 & 681.2 & 679.2 & 683.4 \\
        \bottomrule
    \end{tabular}
    \label{yiduidianzu}
\end{table}
这组电阻的均值为$\overline{R_x} = \frac{1}{8}\sum_{i=1}^{8} R_{i} = \SI{680.3}{\ohm}$,标准偏差为$S = \sqrt{\frac{1}{8-1}\sum_{i=1}^{8}(R_{i} - \overline{R_x})^2} = \SI{2.91}{\ohm}$,则离散度为
\[\frac{S}{\overline{R_x}} \times 100\% = 0.43\% \]


\subsection{误差分析}
\begin{enumerate}
    \item 电阻箱、标准电阻的标称阻值与实际阻值存在偏差,导致测量结果产生系统误差;
    \item 电桥平衡时检流计指针难以精确指零,导致读数误差;
    \item 电阻箱分度值较低,以及电流计的灵敏度过高,无论怎么调整都不能使检流计严格达到零偏,限制了测量结果的准确性;
    \item 个人读数时会出现偏差,尤其是在调节电阻箱时,读数不够精确;
    \item 环境温度变化会影响电阻值,进而影响测量结果。
\end{enumerate}
\subsection{实验探讨}
本次实验使用经典的惠斯登电桥法测量电阻,并使用交换法有效降低误差。
不仅使我对电路原理和实验技巧有了更深刻的认识,
还使我的数据处理、误差分析和有效数字分析的能力得到显著提升。
\section{思考题}
\subsection{为什么用惠斯登电桥测电阻比伏安法测量的准确度高?用电桥法测电阻产生误差的主要因素是什么?}

\subsubsection{准确高的原因}
\begin{enumerate}
   \item 惠斯登电桥基于平衡原理,当电桥平衡时检流计无电流,待测电阻\(R_x\)仅由比率臂\(\frac{R_1}{R_2}\)和标准电阻\(R_s\)决定,无需直接测量电流、电压,避免了伏安法中电表内阻引入的系统误差。
   \item 电桥平衡时,检流计指针指零,读数更为精确,减少了读数误差。
   \item 惠斯登电桥允许通过调整标准电阻\(R_s\)来精确达到平衡状态,提高了测量的灵敏度和准确度。
\end{enumerate}
\subsubsection{用电桥法测电阻产生误差的主要因素}
\begin{enumerate}
   \item 标准电阻\(R_s\)的精度和稳定性不能保证,若其阻值有偏差,会直接影响测量结果。
   \item 检流计灵敏度不足,导致平衡判断存在偏差。
\end{enumerate}
\subsection{为了提高电桥测量灵敏度,应采取哪些措施?为什么?}
\begin{enumerate}
  \item 提高电源电压。原因:电源电压升高会增大桥路电流,使电桥不平衡时检流计偏转幅度更明显,从而提高灵敏度(\(S = \frac{\Delta d}{\frac{\Delta R_s}{R_s}}\))
  \item 使比率臂\(\frac{R_1}{R_2}\)尽可能接近。原因:比率臂接近1时,电桥对电阻变化的灵敏度最高,微小的\(\Delta R_s\)会引发检流计明显偏转。
\end{enumerate}
\subsection{用电桥测电阻时,若线路接通后检流计指针总是往一个方向偏转或总不偏转,试分析是什么原因?}
\subsubsection{检流计总往一个方向偏转}
\begin{enumerate}
   \item 比率臂\(\frac{R_1}{R_2}\)选择不合理,导致\(R_s\)的调节范围远小于(或远大于)\(R_x\),电桥始终无法平衡。
   \item 电路存在短路或断路。
\end{enumerate}
\subsubsection{检流计总不偏转}
\begin{enumerate}
   \item 检流计本身断路或被短路,无法检测电流。
   \item 电源未接通,或桥路中某部分完全断路(如开关未闭合、电阻接线脱落)。
\end{enumerate}
\subsection{惠斯登电桥比率臂选取的原则是什么?为什么要这样选取?}
\subsubsection{选取原则}
\begin{enumerate}
   \item 使比率臂\(\frac{R_1}{R_2}\)尽可能接近1。
   \item 保证标准电阻\(R_s\)的调节能覆盖待测电阻\(R_x\)的范围。
   \item 优先使用电阻箱的高位档位,使\(R_s\)有更多有效数字。
\end{enumerate}

\subsubsection{选取原因}
比率臂接近1时,电桥灵敏度最高,测量误差最小;覆盖\(R_x\)范围可确保电桥能平衡;使用电阻箱高位档位能提高\(R_s\)的测量精度,从而提升\(R_x\)的计算精度。
\subsection{如何使用自组电桥测量电表内阻(注意电表所能允许通过的最大电流)?根据电桥平衡的特点,可否将桥路中的检流计去掉,换成行测电表判别电桥的平衡?}
\subsubsection{测量方法}
\begin{enumerate}
    \item 电路设计:将电流表作为\(R_x\)接入桥臂,串联保护电阻,计算:\(R_{\text{保}}\geq\frac{U}{I_{\text{max}}} - R_{\text{表估}}\),U为电源电压,\(R_{\text{表估}}\)为电表内阻估计值),防止电流过大损坏电表。
    \item 调节平衡:调节\(R_s\)和比率臂\(\frac{R_1}{R_2}\),使检流计指零,此时\(R_x=\frac{R_1}{R_2} \cdot R_s\)。
\end{enumerate}
\subsubsection{能否替换检流计}
理论上可将检流计替换为电压表(测桥路两端电压,电压为零时判断平衡)或电流表(测桥路电流,电流为零时判断平衡),但实际中:
\begin{enumerate}
   \item 电压表内阻并非无穷大,电流表内阻并非零,会改变桥路等效电阻,引入额外误差。
   \item 电流表灵敏度通常低于检流计,难以精确判断平衡状态。
\end{enumerate}
因此,建议仍使用高灵敏度的检流计进行平衡判断,以确保测量精度。

\end{fullreportonly}
\insertnotes
\end{document}
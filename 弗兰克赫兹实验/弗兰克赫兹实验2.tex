\documentclass[]{../template/Report}%方括号内写yuxi即生成预习报告\documentclass[yuxi]{../template/Report}
\usepackage{longtable}
\settemplatedir{../template/}%设置模板路径

\exname{弗兰克赫兹实验} %实验名称
\extable{} %实验桌号
\instructor{} %指导教师
\class{} %班级
\name{} %姓名
\stuid{} %学号

\nyear{} %年
\nmonth{} %月
\nday{} %日
\nweekday{} %星期几,e.g. \nweekday{三}
\daypart{}%上午/下午

\redate{} %如有实验补做,补做日期
\resitu{} %情况说明:

\begin{document}
\maketitle%输出封面

\section{预习报告}
\subsection{实验综述}

根据波尔理论,原子只能处在能量数值分立的某些状态,即能级。原子在能级间进行跃迁时吸收或发射确定频率的光子。
当原子与一定能量的电子发生碰撞时,可使原子从低能级$E_1$跃迁到高能级$E_2$,则有
\[eU = \frac{1}{2} m_e v^2 = E_2 - E_1\]
原子从基态跃迁到第一激发态的$U_1$称为原子第一激发电势。本实验将测量这个数值。

本实验原理图如\cref{yuanlitu}所示。
\begin{figure}[htbp]
    \centering
    \includegraphics[width=0.4\textwidth]{figures/yuanlitu.png}
    \caption{弗兰克赫兹实验原理图}
    \label{yuanlitu}
\end{figure}
弗兰克赫兹管是一个具有双栅极结构的充氩四极电子管。
在电压$U_F$作用下,阴极$F$发出电子,并经电场$U_{G_2K}$加速趋向阳极$A$,在电流计中产生电流。
第一栅极电压$U_{G_1K}$用于控制电子流大小,并抵消阴极$K$附近电子云形成负电压的影响。
第二栅极电压$U_{G_2K}$用于加速电子。
拒斥电压$U_{G_2A}$用于筛选能量大于$U_{G_2A}$的电子到达阳极$A$。

从$0$开始增大$U_{G_2K}$。$U_{G_2K} < U_1$时,电子能量不足以令能级跃迁,只与氩原子弹性碰撞。$U_{G_2K}$越大,阳极的电流就越大。
当$U_{G_2K} = U_1$时,电子能量可使氩原子能级跃迁,将把全部动能传给氩原子,由于$U_{G_2A}$作用,失去能力的电子不能达到阳极$A$,电流计的电流$I_A$第一次大幅度下降。
随着$U_{G_1K}$增加,电子与原子发生非弹性碰撞的区域向$G_1$移动,在$G_2$附近电子重新获得的能力小于$U_1$,不会再发生非弹性碰撞,电子将克服$U_{G_2A}$到达阳极,$I_A$又开始增加了。
以此类推,如\cref{example_figure},每当$U_{G_2K} = n U_1$时,$I_A$都将大幅下降一次,$I_A \sim U_{G_2K}$图线将不断出现峰和谷,并且相邻峰(或谷)间距皆相等,约为$U_1$。
\begin{figure}[htbp]
    \centering
    \includegraphics[width=0.6\textwidth]{figures/example_figure.png}
    \caption{理想的$I_A \sim U_{G_2K}$曲线}
    \label{example_figure}
\end{figure}

本实验分以下两部分。
\subsubsection{测量氩原子的第一激发电势}
\begin{enumerate}
    \item 调试好仪器。连接和预热设备。将示波器CH1通道设置为$\SI{200}{mV/div}, \SI{500}{\mu s/div}$。
    外触发通道的触发类型为边沿出发,触发源为外触发输入通道,下降沿触发,触发方式为自动。
    \item 令“电源二”实验仪处于自动测量状态。按“确认”按键,启动自动测量,在示波器上观测波浪式爬坡曲线的形成过程。约4分钟之后,自动测量过程结束,记录$I_A \sim U_{G_2K}$曲线各峰值序号和加速电压。
    \item 重复测量6次,并用逐差法或最小二乘法等计算氩原子的第一激发电势。
\end{enumerate}

\subsubsection{测量$I_A \sim U_{G_2K}$曲线}
\begin{enumerate}
    \item 调试好仪器,确认“电源二”实验仪处于手动测量状态。
    \item 从$\SI{0.0}{V}$到$\SI{100.0}{V}$,缓慢调节加速电压$U_{G_2K}$,每间隔$\SI{0.5}{V}$记录一次$I_A$,在峰谷值附近可多测几组数据。
    \item 绘制$I_A \sim U_{G_2K}$曲线,并根据数据得出氩原子的第一激发电势。
\end{enumerate}


\subsection{实验重点}
\begin{enumerate}
    \item 验证原子能级的存在,深入理解“量子化”概念;
    \item 学习仪器使用和原子第一激发电势的测量方法;
    \item 掌握大量数据的处理技巧。
\end{enumerate}

\subsection{实验难点}
\begin{enumerate}
    \item 初次接触仪器,使用上可能有困难;
    \item 本实验要求记录和处理大量数据,需要一定技巧;
    \item 能够利用最小二乘法等计算第一激发电势。
\end{enumerate}

\begin{fullreportonly}
% \section{原始数据}
% 如\cref{datas}所示。
% \begin{figure}[htbp]
%     \centering
%     \begin{subfigure}[b]{0.48\textwidth}
%         \includegraphics[width=\textwidth]{data1.jpg}
%         \caption{多量程电流表设计电路}
%     \end{subfigure}
%     \hfill
%     \begin{subfigure}[b]{0.48\textwidth}
%         \includegraphics[width=\textwidth]{data2.jpg}
%         \caption{多量程电流表校验电路}
%     \end{subfigure}
%     \caption{原始数据}
%     \label{datas}
% \end{figure}

\section{结果与分析}
\subsection{数据处理与结果}
\subsubsection{自动测量氩原子的第一激发电势}
调节仪器参数$U_{F_1F_2} = \SI{3.53}{V}, U_{G_1K} = \SI{1.25}{V}, U_{G_2A} = \SI{2.49}{V}$,微电流计量程为$1 \times \SI{e-8}{A}$。自动测量五次,测量结果如\cref{zidongceliang}。
\begin{table}[htbp]
  \centering
  \caption{自动测量的实验数据}
  \label{zidongceliang}
  \begin{tabular}{C{.08\textwidth}C{.15\textwidth}C{.08\textwidth}C{.08\textwidth}C{.08\textwidth}C{.08\textwidth}C{.08\textwidth}C{.08\textwidth}C{.08\textwidth}}
    \toprule
    \multirow{3}{*}{1} & 峰值序号 & 1 & 2 & 3 & 4 & 5 & 6 & 7 \\
    \cmidrule{2-9}
    & $U_{G_1K} / \si{V}$ & 16.1 & 27.0 & 37.7 & 49.9 & 61.5 & 74.4 & 87.1 \\
    \cmidrule{2-9}
    & $I_A/(\times \SI{e-8}{A})$ & 55 & 67 & 79 & 93 & 108 & 128 & 156 \\
    \midrule

    \multirow{3}{*}{2} & 峰值序号 & 1 & 2 & 3 & 4 & 5 & 6 & 7 \\
    \cmidrule{2-9}
    & $U_{G_1K} / \si{V}$ & 16.5 & 27.4 & 38.5 & 49.5 & 61.7 & 74.1 & 87.3 \\
    \cmidrule{2-9}
    & $I_A/(\times \SI{e-8}{A})$ & 53 & 65 & 77 & 90 & 105 & 124 & 152 \\
    \midrule

    \multirow{3}{*}{3} & 峰值序号 & 1 & 2 & 3 & 4 & 5 & 6 & 7 \\
    \cmidrule{2-9}
    & $U_{G_1K} / \si{V}$ & 15.0 & 27.2 & 38.2 & 49.5 & 61.4 & 74.3 & 87.3 \\
    \cmidrule{2-9}
    & $I_A/(\times \SI{e-8}{A})$ & 53 & 66 & 78 & 91 & 106 & 126 & 154 \\
    \midrule

    \multirow{3}{*}{4} & 峰值序号 & 1 & 2 & 3 & 4 & 5 & 6 & 7 \\
    \cmidrule{2-9}
    & $U_{G_1K} / \si{V}$ & 15.4 & 26.8 & 39.6 & 49.5 & 61.5 & 74.3 & 87.3 \\
    \cmidrule{2-9}
    & $I_A/(\SI{e-8}{A})$ & 54 & 66 & 79 & 92 & 107 & 127 & 155 \\
    \midrule

    \multirow{3}{*}{5} & 峰值序号 & 1 & 2 & 3 & 4 & 5 & 6 & 7 \\
    \cmidrule{2-9}
    & $U_{G_1K} / \si{V}$ & 15.1 & 27.2 & 38.1 & 49.3 & 62.0 & 74.7 & 87.3 \\
    \cmidrule{2-9}
    & $I_A/(\SI{e-8}{A})$ & 54 & 67 & 79 & 92 & 108 & 128 & 156 \\ 
    \bottomrule
  \end{tabular}
\end{table}

对每次实验,利用Origin拟合$U_{G_1K} \sim $峰值序号图线,斜率即为第一激发电势的测量值。分别是:

\begin{center}
    $U_1^{(1)} = \SI{11.84}{V}$ \\ % 0.17894
    $U_1^{(2)} = \SI{11.75}{V}$    \\ % 0.19233
    $U_1^{(3)} = \SI{11.94}{V}$ \\ % 0.15958
    $U_1^{(4)} = \SI{11.91}{V}$ \\ % 0.15545
    $U_1^{(5)} = \SI{11.98}{V}$ \\ % 0.15504
\end{center}

取平均得氩原子的第一激发电势的测量值为$\bar{U_1} =  \SI{11.89}{V}$。

\subsubsection{手动测量$I_A \sim U_{G_2K}$曲线}

调节仪器参数$U_{F_1F_2} = \SI{3.53}{V}, U_{G_1K} = \SI{1.25}{V}, U_{G_2A} = \SI{2.49}{V}$,微电流计量程为$1\times \SI{10e-8}{A}$。手动测量数据如\cref{shoudongceliang}所示。
\begin{longtable}{C{.10\textwidth}*{10}{C{.06\textwidth}}}
    \caption{手动测量的$I_A \sim U_{G_2K}$数据}
    \label{shoudongceliang}\\
    \toprule
    $U_{G_2K}(\si{V})$ & 0.50 & 1.00 & 1.50 & 2.00 & 2.50 & 3.00 & 3.50 & 4.00 & 4.50 & 5.00 \\
    \midrule
    $I_A(\times 10^{-8}\si{A})$ & 0.00 & 0.00 & 0.00 & 0.00 & 0.00 & 0.00 & 2.00 & 5.00 & 10.00 & 17.00 \\
    \midrule
    $U_{G_2K}(\si{V})$ & 5.50 & 6.00 & 6.50 & 7.00 & 7.50 & 8.00 & 8.50 & 9.00 & 9.50 & 10.00 \\
    \midrule
    $I_A(\times 10^{-8}\si{A})$ & 24.00 & 30.00 & 36.00 & 39.00 & 42.00 & 43.00 & 45.00 & 46.00 & 47.00 & 47.00 \\
    \midrule
    $U_{G_2K}(\si{V})$ & 10.50 & 11.00 & 11.50 & 12.00 & 12.50 & 13.00 & 13.50 & 14.00 & 14.50 & 15.00 \\
    \midrule
    $I_A(\times 10^{-8}\si{A})$ & 48.00 & 49.00 & 49.00 & 50.00 & 50.00 & 51.00 & 51.00 & 52.00 & 52.00 & 53.00 \\
    \midrule
    $U_{G_2K}(\si{V})$ & 15.50 & 16.00 & 16.50 & 17.00 & 17.50 & 18.00 & 18.50 & 19.00 & 19.50 & 20.00 \\
    \midrule
    $I_A(\times 10^{-8}\si{A})$ & 53.00 & 53.00 & 54.00 & 54.00 & 53.00 & 52.00 & 51.00 & 49.00 & 48.00 & 47.00 \\
    \midrule
    $U_{G_2K}(\si{V})$ & 20.50 & 21.00 & 21.50 & 22.00 & 22.50 & 23.00 & 23.50 & 24.00 & 24.50 & 25.00 \\
    \midrule
    $I_A(\times 10^{-8}\si{A})$ & 47.00 & 49.00 & 50.00 & 52.00 & 54.00 & 56.00 & 58.00 & 59.00 & 61.00 & 62.00 \\
    \midrule
    $U_{G_2K}(\si{V})$ & 25.50 & 26.00 & 26.50 & 27.00 & 27.50 & 28.00 & 28.50 & 29.00 & 29.50 & 30.00 \\
    \midrule
    $I_A(\times 10^{-8}\si{A})$ & 64.00 & 65.00 & 65.00 & 66.00 & 66.00 & 66.00 & 66.00 & 64.00 & 62.00 & 59.00 \\
    \midrule
    $U_{G_2K}(\si{V})$ & 30.50 & 31.00 & 31.50 & 32.00 & 32.50 & 33.00 & 33.50 & 34.00 & 34.50 & 35.00 \\
    \midrule
    $I_A(\times 10^{-8}\si{A})$ & 55.00 & 52.00 & 50.00 & 50.00 & 52.00 & 55.00 & 59.00 & 62.00 & 65.00 & 69.00 \\
    \midrule
    $U_{G_2K}(\si{V})$ & 35.50 & 36.00 & 36.50 & 37.00 & 37.50 & 38.00 & 38.50 & 39.00 & 39.50 & 40.00 \\
    \midrule
    $I_A(\times 10^{-8}\si{A})$ & 71.00 & 74.00 & 76.00 & 78.00 & 79.00 & 79.00 & 80.00 & 80.00 & 79.00 & 78.00 \\
    \midrule
    $U_{G_2K}(\si{V})$ & 40.50 & 41.00 & 41.50 & 42.00 & 42.50 & 43.00 & 43.50 & 44.00 & 44.50 & 45.00 \\
    \midrule
    $I_A(\times 10^{-8}\si{A})$ & 76.00 & 72.00 & 68.00 & 63.00 & 58.00 & 55.00 & 54.00 & 57.00 & 61.00 & 66.00 \\
    \midrule
    $U_{G_2K}(\si{V})$ & 45.50 & 46.00 & 46.50 & 47.00 & 47.50 & 48.00 & 48.50 & 49.00 & 49.50 & 50.00 \\
    \midrule
    $I_A(\times 10^{-8}\si{A})$ & 71.00 & 76.00 & 81.00 & 84.00 & 87.00 & 90.00 & 92.00 & 93.00 & 94.00 & 94.00 \\
    \midrule
    $U_{G_2K}(\si{V})$ & 50.50 & 51.00 & 51.50 & 52.00 & 52.50 & 53.00 & 53.50 & 54.00 & 54.50 & 55.00 \\
    \midrule
    $I_A(\times 10^{-8}\si{A})$ & 94.00 & 93.00 & 92.00 & 90.00 & 87.00 & 82.00 & 76.00 & 70.00 & 66.00 & 64.00 \\
    \midrule
    $U_{G_2K}(\si{V})$ & 55.50 & 56.00 & 56.50 & 57.00 & 57.50 & 58.00 & 58.50 & 59.00 & 59.50 & 60.00 \\
    \midrule
    $I_A(\times 10^{-8}\si{A})$ & 66.00 & 69.00 & 74.00 & 80.00 & 85.00 & 90.00 & 95.00 & 100.00 & 103.00 & 106.00 \\
    \midrule
    $U_{G_2K}(\si{V})$ & 60.50 & 61.00 & 61.50 & 62.00 & 62.50 & 63.00 & 63.50 & 64.00 & 64.50 & 65.00 \\
    \midrule
    $I_A(\times 10^{-8}\si{A})$ & 108.00 & 110.00 & 111.00 & 111.00 & 111.00 & 110.00 & 109.00 & 106.00 & 102.00 & 97.00 \\
    \midrule
    $U_{G_2K}(\si{V})$ & 65.50 & 66.00 & 66.50 & 67.00 & 67.50 & 68.00 & 68.50 & 69.00 & 69.50 & 70.00 \\
    \midrule
    $I_A(\times 10^{-8}\si{A})$ & 92.00 & 87.00 & 83.00 & 82.00 & 83.00 & 86.00 & 90.00 & 95.00 & 101.00 & 106.00 \\
    \midrule
    $U_{G_2K}(\si{V})$ & 70.50 & 71.00 & 71.50 & 72.00 & 72.50 & 73.00 & 73.50 & 74.00 & 74.50 & 75.00 \\
    \midrule
    $I_A(\times 10^{-8}\si{A})$ & 111.00 & 116.00 & 120.00 & 124.00 & 126.00 & 129.00 & 130.00 & 132.00 & 132.00 & 132.00 \\
    \midrule
    $U_{G_2K}(\si{V})$ & 75.50 & 76.00 & 76.50 & 77.00 & 77.50 & 78.00 & 78.50 & 79.00 & 79.50 & 80.00 \\
    \midrule
    $I_A(\times 10^{-8}\si{A})$ & 131.00 & 129.00 & 126.00 & 122.00 & 117.00 & 113.00 & 110.00 & 108.00 & 108.00 & 110.00 \\
    \midrule
    $U_{G_2K}(\si{V})$ & 80.50 & 81.00 & 81.50 & 82.00 & 82.50 & 83.00 & 83.50 & 84.00 & 84.50 & 85.00 \\
    \midrule
    $I_A(\times 10^{-8}\si{A})$ & 112.00 & 116.00 & 120.00 & 126.00 & 130.00 & 135.00 & 140.00 & 145.00 & 149.00 & 153.00 \\
    \midrule
    $U_{G_2K}(\si{V})$ & 85.50 & 86.00 & 86.50 & 87.00 & 87.50 & 88.00 & 88.50 & 89.00 & 89.50 & 90.00 \\
    \midrule
    $I_A(\times 10^{-8}\si{A})$ & 156.00 & 158.00 & 160.00 & 162.00 & 162.00 & 162.00 & 160.00 & 158.00 & 155.00 & 152.00 \\
    \midrule
    $U_{G_2K}(\si{V})$ & 90.50 & 91.00 & 91.50 & 92.00 & 92.50 & 93.00 & 93.50 & 94.00 & 94.50 & 95.00 \\
    \midrule
    $I_A(\times 10^{-8}\si{A})$ & 150.00 & 147.00 & 147.00 & 147.00 & 148.00 & 150.00 & 153.00 & 157.00 & 161.00 & 165.00 \\
    \midrule
    $U_{G_2K}(\si{V})$ & 95.50 & 96.00 & 96.50 & 97.00 & 97.50 & 98.00 & 98.50 & 99.00 & 99.50 & 100.00 \\
    \midrule
    $I_A(\times 10^{-8}\si{A})$ & 170.00 & 176.00 & 181.00 & 186.00 & 191.00 & 195.00 & 199.00 & 203.00 & 206.00 & 208.00 \\
    \bottomrule
\end{longtable}


利用Origin绘制出$I_A \sim U_{G_2K}$曲线如\cref{shoudonggraph}所示。
\begin{figure}[htbp]
    \centering
    \includegraphics[width=0.6\textwidth]{shoudonggraph.png}
    \caption{手动测量的$I_A \sim U_{G_2K}$曲线}
    \label{shoudonggraph}
\end{figure}

Origin读取到$U_{G_1K}$的6个峰(自动弃去第一个峰值不明显的峰),分别是$\SI{21.11}{V}$,$\SI{32.29}{V}$,$\SI{43.69}{V}$,$\SI{55.39}{V}$,$\SI{67.41}{V}$,$\SI{79.635}{V}$。
同样按最小二乘法计算得$\bar{U_1}=\SI{11.71}{V}$。

\subsection{误差分析}
\subsubsection{自动测量氩原子的第一激发电势}
本部分实验A类不确定度来源有两个,重复实验的散布,和单次最小二乘拟合的不确定度。

第一部分不确定度,即各$U_1$测量值对平均值的标准误:
\[u_{A1} = \sqrt{\frac{1}{5(5-1)} \sum_{i=1}^{5} \left(U_1^{(i)} - \bar{U_1}\right)^2} = 0.04\]

从Origin读取各次试验最小二乘拟合的标准偏差分别为$\SI{0.17894}{V}$,$\SI{0.19233}{V}$,$\SI{0.15958}{V}$,$\SI{0.15545}{V}$,$\SI{0.15504}{V}$。
因此第二部分不确定度,即各次拟合相对不确定度的均方根:
\[u_{A2} = \bar{U_1} \sqrt{\frac{1}{5} \sum_{i=1}^{5} \left(\frac{s_i}{U_1^{(i)}}\right)^2} = \SI{0.17}{V}\]。

合成得A类不确定度为:
\[u_A = \sqrt{u_{A1}^2 + u_{A2}^2} = \SI{0.17}{V}\]

根据仪器说明书,仪器精度为$\pm 1\%$。因此B类不确定度为:
\[u_B = \frac{\Delta_\text{仪}}{\sqrt{3}} = \SI{0.0006}{V}\]

因此合成不确定度为:
\[u = \sqrt{u_A^2 + u_B^2} = \SI{0.17}{V}\]

最终计算结果为:
\[U_1 = (11.89 \pm 0.17)\si{V}\]

与标准值$\SI{11.61}{V}$的相对误差为:
\[\delta = \frac{\left|\bar{U_1} - \tilde{U_1}\right|}{\tilde{U_1}} \times 100 \% = 2.41\%\]

这部分实验的误差来源主要有:
\begin{enumerate}
    \item 第一个峰的峰值不明显,导致仪器自动测量结果与理论有偏差;
    \item 建立的物理模型和实际不一致,包括存在接触电势差,能量传递过程不能等效为一维碰撞等;
    \item 实验仪器精度有限,读取电流有效位数较小,难以准确判别峰值;
    \item 仪器特性与理想情况不一致,可能存在老化、噪声等情况;
    \item 温度等环境因素的影响。
\end{enumerate}

\subsubsection{手动测量并计算氩原子的第一激发电势}

根据Origin最小二乘拟合结果,A类不确定度,也即标准偏差为:
\[u_A = \SI{0.10}{V}\]

同理B类不确定度为:
\[u_B = \frac{\Delta_\text{仪}}{\sqrt{3}} = \SI{0.0006}{V}\]

合成不确定度为:
\[u = \sqrt{u_A^2 + u_B^2} = \SI{0.10}{V}\]

最终计算结果为:
\[U_1 = (11.71 \pm 0.10)\si{V}\]

与标准值$\SI{11.61}{V}$的相对误差为:
\[\delta = \frac{\left|\bar{U_1} - \tilde{U_1}\right|}{\tilde{U_1}} \times 100 \% = 0.86\%\]

这部分实验的误差来源主要有:
\begin{enumerate}
    \item 测量次数只有一次,可能存在偶然误差;
    \item 建立的物理模型和实际不一致,包括存在接触电势差,能量传递过程不能等效为一维碰撞等;
    \item 实验仪器精度有限,读取电流有效位数较小,难以准确判别峰值;
    \item 仪器特性与理想情况不一致,可能存在老化、噪声等情况;
    \item 温度等环境因素的影响。
\end{enumerate}

\subsection{实验探讨}

弗兰克赫兹实验通过将氩原子的第一激发电势转为栅极电压,验证了原子能级的量子化。仪器测量部分让我感到自动化实验的魅力,手动测量部分使我增强了数据处理的本领。也使我对误差的分析和计算更为熟练。

\section{思考题}
\subsection{原子第一激发态的物理意义是什么?}
即电子从基态跃迁到能量最低的激发能级时随处的状态。说明了原子的能级是分立的。
在本实验中,动能大于氩原子第一激发态和基态间能级差的电子,能够与氩原子碰撞后将其激发到第一激发态,从而令$I_A \sim U_{G_2K}$产生一组组谷。

\subsection{温度对$I \sim U $曲线的影响?}
\begin{enumerate}
    \item 曲线整体上移。因为阴极温度升高,电子热发射电流增强,进入管内的电子数目增加,整体电流水平上升;
    \item 曲线相应峰谷间电流差减小。因为电子初始能量分布展宽,使得$I \sim U$曲线逐渐变得平滑。
\end{enumerate}
\end{fullreportonly}
\insertnotes
\end{document}
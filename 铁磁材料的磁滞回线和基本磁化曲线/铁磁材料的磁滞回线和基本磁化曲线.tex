\documentclass[yuxi]{../template/Report}%方括号内写yuxi即生成预习报告
\settemplatedir{../template/}%设置模板路径

\exname{~~~~~~铁磁材料的磁滞回线和基本磁化曲线} %实验名称
\extable{} %实验桌号
\instructor{张建华} %指导教师
\class{信工2401} %班级
\name{姚舜瑜} %姓名
\stuid{3240100532} %学号

\nyear{2025} %年
\nmonth{11} %月
\nday{11} %日
\nweekday{二} %星期几,e.g. \nweekday{三}
\daypart{上午}%上午/下午

\redate{} %如有实验补做,补做日期
\resitu{} %情况说明:

\begin{document}
\maketitle%输出封面

\section{预习报告(10分)}

\subsection{实验综述(5分)}
\subsubsection{实验目的}
\begin{enumerate}
    \item 了解采用示波器测动态磁滞回线的原理,学会用示波器测绘基本磁化曲线和磁滞回线。
    \item 掌握磁滞、磁滞回线和磁化曲线的基本概念,测量材料的饱和磁感应强度 $B_s$、剩磁 $B_r$、矫顽力 $H_c$ 的数值。
    \item 深刻理解物理实验中的转换法
\end{enumerate}
\subsubsection{实验原理}
\noindent\textbf{1. 背景与基本关系}

铁磁材料内部存在大量自发磁化的磁畴,外加磁场使磁畴壁运动与磁矩转向,从而产生强烈而非线性的磁响应。宏观上用磁场强度 $H$、磁感应强度 $B$ 与磁化强度 $M$ 描述材料磁状态,基本关系为
\begin{equation}
    B = \mu_0 (H + M),\quad (\mu_0=4\pi\times10^{-7}\,\mathrm{H/m}) ,
\end{equation}
其中 $M$ 随 $H$ 的变化具有明显非线性与滞后。典型特征量包括:饱和磁感应强度 $B_s$($H$ 充分大时 $B$ 的极限值)、剩磁 $B_r$(撤去外场时 $B$ 的截距)、矫顽力 $H_c$(将 $B$ 退回到零所需的反向场)。磁滞回线所围成的面积满足
\begin{equation}
    w = \oint H\,\mathrm{d}B\quad (\mathrm{J/m^3\,per\,cycle}),
\end{equation}
表示每周期的磁滞能量损耗密度。

\vspace{0.5em}
\noindent\textbf{2. 铁磁材料的磁化过程简述}

随外场从零逐渐增加:先以磁畴壁位移为主(阻力小,$B$ 上升快),随后转为畴内磁矩旋转;当大部分磁矩沿场定向时进入饱和区。撤场后由于缺陷、内应力与磁各向异性等钉扎效应,体系不能沿原路返回,形成磁滞。

\vspace{0.5em}
\noindent\textbf{3. 磁滞回线与基本磁化曲线}

完整的 $B$–$H$ 闭合曲线即磁滞回线。若试样先退磁至无偏置(平均 $M\!\approx\!0$),再从小场开始逐步增大至饱和,所得到的从原点出发并逐渐上卷到饱和的包络曲线称为\emph{基本磁化曲线}(或初始磁化曲线)。在动态法中可通过多组小幅度交流“次回线”的顶点轨迹得到该曲线(见第5点)。

\vspace{0.5em}
\noindent\textbf{4. 动态法(示波法)测磁滞回线的电路等效}

实验常用环形(闭合磁路)试样,平均磁路长度 $l_m$,截面积 $S$。在试样上绕两组线圈:原边励磁绕组 $N_1$ 匝与副边测量绕组 $N_2$ 匝。
\begin{itemize}
    \item 励磁回路串联取样电阻 $R_s$。励磁电流 $i_1$ 近似与 $H$ 成正比:
    \begin{equation}
        H = \frac{N_1 i_1}{l_m} = \frac{N_1}{l_m R_s}\,u_{R_s} .
    \end{equation}
    于是可将 $u_{R_s}$ 送入示波器 $X$ 轴,得到 $H$ 的电压表征。
    \item 副边线圈感应电动势 $u_2$ 与 $B$ 的变化率相关:
    \begin{equation}
        u_2 = -N_2 S\,\frac{\mathrm{d}B}{\mathrm{d}t} .
    \end{equation}
    为得到与 $B$ 成正比的量,将 $u_2$ 送入被动积分网络 $R_i$–$C$。课件条件写作 $R_2 \gg 1/(\omega C)$,与本式 $\omega R_i C\!\gg\!1$ 等价;在该条件下,电容两端电压近似为输入的时间积分:
    \begin{equation}
        u_C \approx \frac{1}{R_i C}\int u_2\,\mathrm{d}t = -\frac{N_2 S}{R_i C}\,B + \text{const}.\label{eq:int}
    \end{equation}
    取交流稳态并消除直流常数后,有 $|B| \propto |u_C|$(极性取决于接线方向)。课件中的关系 $U_Y = (N_2 S/(C R_2))\,B$ 与此相符(忽略符号)。因此将 $u_C$ 送入示波器 $Y$ 轴,得到 $B$ 的电压表征。
\end{itemize}
由此建立了电—磁量的\emph{转换法}:
\begin{equation}
    H = K_H\,u_X,\quad K_H = \frac{N_1}{l_m R_s};\qquad
    B = K_B\,u_Y,\quad K_B = -\frac{R_i C}{N_2 S},\label{eq:kbkh}
\end{equation}
其中 $u_X$、$u_Y$ 分别为输入示波器 $X$、$Y$ 通道的电压。将示波器置 $X$–$Y$ 模式,即可直接显示 $B$–$H$ 曲线。

\vspace{0.25em}
\emph{与课件符号的快速对照:} $R_1\!(\text{课件})\equiv R_s\!(\text{本文})$,$R_2\equiv R_i$,$L\,(\text{课件图示中的磁路长度})\equiv l_m$。励磁侧有 $H = N_1 I_1/l_m$,而 $U_X = I_1 R_1$,故 $H \propto U_X$;测量侧积分有 $U_Y \propto B$。

\vspace{0.5em}
\noindent\textbf{5. 动态法测基本磁化曲线的实现}

先对试样退磁(加交流并缓慢衰减到零,减小残余偏置)。随后以正弦励磁,从很小幅值开始逐步增大。每一幅值对应一个以原点为中心的“次回线”,其最外侧顶点($H$ 达到该幅值的极值时的 $B$)构成的轨迹即初始磁化曲线。记录若干幅值下顶点坐标 $(H_{\max},B)$,即可拟合出从近似线性区到饱和区的基本磁化曲线。

为与课件记号一致,初始磁化曲线常记作 $O s$(或 $O\!S$)曲线;完整磁滞回线的典型路径顺序为 $o\!\to a\to b\to c\to s$(正向饱和)再经 $d\to e\to s'$ 返回,对应去磁—反向磁化—反向去磁等阶段。

\vspace{0.5em}
\noindent\textbf{6. 用示波器读出 $B_s$、$B_r$、$H_c$ 的换算关系}

设示波器灵敏度为 $s_X$(V/格)、$s_Y$(V/格),读数为 $x$、$y$ 格,则有
\begin{equation}
    u_X = s_X\,x,\qquad u_Y = s_Y\,y.
\end{equation}
由式\eqref{eq:kbkh} 可得
\begin{equation}
    H = K_H s_X\,x = \frac{N_1}{l_m R_s}\,s_X\,x,\qquad
    B = K_B s_Y\,y = -\frac{R_i C}{N_2 S}\,s_Y\,y.
\end{equation}
据此:
\begin{itemize}
    \item 矫顽力 $H_c$:回线与 $B=0$ 轴的两个交点的横坐标模值,$H_c = \frac{N_1}{l_m R_s}\,s_X\,|x_{B=0}|$。
    \item 剩磁 $B_r$:回线与 $H=0$ 轴的两个交点的纵坐标模值,$B_r = \frac{R_i C}{N_2 S}\,s_Y\,|y_{H=0}|$(取正值)。
    \item 饱和磁感应强度 $B_s$:$Y$ 方向达到平台时的极值,$B_s = \frac{R_i C}{N_2 S}\,s_Y\,|y_{\max}|$。
\end{itemize}

\vspace{0.5em}
\noindent\textbf{7. 软磁与硬磁材料(与课件3.3对齐)}

软磁材料具有高磁导率 $\mu$、低剩磁 $B_r$、低矫顽力 $H_c$(典型 $\sim10^0\!\text{--}\!10^1\,\mathrm{A/m}$),磁滞回线“瘦”,损耗小,常用于变压器、电机铁芯等需频繁磁化反转的场合;硬磁材料具有较高的 $B_r$ 与 $H_c$(典型 $\sim10^4\!\text{--}\!10^6\,\mathrm{A/m}$),回线“胖”,适合作为永磁体(电表、扬声器等)。磁滞损耗与回线面积成正比(每周能量密度 $w = \oint H\,\mathrm{d}B$)。

\vspace{0.5em}
\noindent\textbf{8. 积分网络工作条件与误差要点}

为使式\eqref{eq:int} 近似成立,应选 $\omega R_i C\!\gg\!1$(例如 $\omega R_i C \ge 50$),并使积分网络不过载($u_C$ 不削顶)。主要误差来源包括:$R_s,\,R_i,\,C$ 标称与温漂、$N_1,\,N_2$ 匝数误差、$l_m,\,S$ 的几何测量误差、示波器刻度与失真、剩余气隙导致的磁路不闭合等。环形样品可显著减小漏磁与端部效应。

\vspace{0.5em}
\noindent\textbf{9. 示波器 $X$–$Y$ 模式与李萨如图(与课件3.4补充)}

工作在 $X$–$Y$ 模式时,关闭时基扫线,用 CH1/CH2 分别驱动 $X/Y$ 偏转。若两通道频率相同且相位同步,则显示 $B$–$H$ 曲线;若存在相位差,会出现李萨如图形(如“8”字形),应检查通道耦合(设为 DC)、探头衰减与极性、触发与增益匹配,以避免相位失配造成读数偏差。

\vspace{0.5em}
\noindent\textbf{10. 实验要点小结}

通过将不可直接观测的 $H$、$B$ 分别转换为易测电压 $u_X$、$u_Y$,在示波器 $X$–$Y$ 模式下即可动态显示磁滞回线,并利用刻度换算得到 $B_s$、$B_r$、$H_c$ 与基本磁化曲线。这正是物理实验中常用的“\emph{转换法}”的典型应用。

\subsection{实验重点(3分)}
(简述本实验的学习重点,不超过100字。)

\subsection{实验难点(2分)}
(简述本实验的实现难点,不超过100字。)

\begin{fullreportonly}
\section{原始数据(20分)}
(将有老师签名的“自备数据记录草稿纸”的扫描或手机拍摄图粘贴在下方,完整保留姓名,学号,教师签字和日期。)

\section{结果与分析(60分)}
\subsection{数据处理与结果(30分)}
(列出数据表格、选择适合的数据处理方法、写出测量或计算结果。)

\subsection{误差分析(20分)}
(运用测量误差、相对误差或不确定度等分析实验结果,写出完整的结果表达式,并分析误差原因。)

\subsection{实验探讨(10分)}
(对实验内容、现象和过程的小结,不超过100字。)

\section{思考题(10分)}
(解答教材或讲义或老师布置的思考题,请先写题干,再作答。)
\end{fullreportonly}
\insertnotes
\end{document}
\documentclass[]{../template/Report}%方括号内写yuxi即生成预习报告\documentclass[yuxi]{../template/Report}
\settemplatedir{../template/}%设置模板路径

\exname{万用表的设计} %实验名称
\extable{7} %实验桌号
\instructor{居乐乐} %指导教师
\class{信工2401} %班级
\name{姚舜瑜} %姓名
\stuid{3240100532} %学号

\nyear{2025} %年
\nmonth{11} %月
\nday{4} %日
\nweekday{二} %星期几,e.g. \nweekday{三}
\daypart{上午}%上午/下午

\redate{} %如有实验补做,补做日期
\resitu{} %情况说明:

\begin{document}
\maketitle%输出封面

\section{预习报告(10分)}
\subsection{实验综述(5分)}
\subsubsection{预备实验:测量电流计的内阻}
在本预备实验中,我们将使用替代法和中值法测量电流计的内阻。
\paragraph{替代法步骤如下:}
\begin{enumerate}
    \item 闭合开关 K,通过调节变阻器 $R$,使标准电流计指针偏转到满刻度(最大电流 $I_g$)。此时记录下变阻器的阻值 $R$。
    \item 断开待测电流计,接入电阻箱 $R_T$,调节 $R_T$ 使标准电流计指针再次偏转到满刻度($I_g$)。
    \item 此时电阻箱的阻值 $R_T$便是待测电流计的内阻。 
\end{enumerate}
\paragraph{中值法步骤如下:}
\begin{enumerate}
    \item 闭合开关 K,调节变阻器 $R$,使电流计指针偏转到满刻度(最大电流 $I_g$)。
    \item 保持 $R$ 的阻值不变,将电阻箱 $R_Z$ 并联在电流计两端,调节 $R_Z$ 使电流计指针退回至满刻度的一半,即 $I_g/2$。
    \item 此时电阻箱的阻值 $R_Z$便是待测电流计的内阻。
\end{enumerate}


\begin{table}[H]
    \centering
    \begin{minipage}{0.4\textwidth}
    \includegraphics[width=\textwidth]{figures/替代法.png}
    \label{fig:替代法}
    \subcaption{替代法测量电流计内阻示意图}
    \end{minipage}
    \begin{minipage}{0.4\textwidth}
    \includegraphics[width=\textwidth]{figures/中值法.png}
    \label{fig:中值法}
    \subcaption{中值法测量电流计内阻示意图}
    \end{minipage}
    \caption{替代法与中值法测量电流计内阻示意图}
\end{table}

\subsubsection{设计、改装并校准电流表}
本实验要求设计一个改装电流表的电路,以实现特定的量程。通过计算所需的分流电阻值,并使用电阻箱进行调节,最终校准改装后的电流表以确保其读数准确。

要实现一个新的量程 $I_{new}$,我们需要给电流计并联一个分流电阻 $R_{1}$,其阻值的计算公式为:
\begin{equation}
    R_{1} = \frac{R_{g} I_{g}}{I_{new} - I_{g}}
\end{equation}
其中,$R_{g}$ 是电流表的内阻,$I_{g}$ 是电流表的满刻度电流。
因此我们设计了如\ref{fig:电流表改装电路}的改装电路,将分流电阻 $R_{1}$ 与电流表并联连接。

接下来,我们需要对改装后的电流表进行校准。使用\ref{fig:电流表校准电路}所示电路校验并校准改装电流表,若测量值偏小,则增大电阻箱内阻,反之则减小电阻箱内阻,直至测量值与标准电流表读数相符。记录所得数据并分析误差。

\begin{table}[H]
    \centering
    \begin{minipage}{0.4\textwidth}
    \includegraphics[width=\textwidth]{figures/gaizhuangA.png}
    \subcaption{电流表改装电路示意图}
    \label{fig:电流表改装电路}
    \end{minipage}
    \begin{minipage}{0.4\textwidth}
    \includegraphics[width=\textwidth]{figures/jiaozhunA.png}
    \subcaption{电流表校准电路示意图}
    \label{fig:电流表校准电路}
    \end{minipage}
    \caption{电流表改装与校准示意图}
\end{table}

\subsubsection{设计、改装并校准电压表}
本实验要求设计一个改装电压表的电路,以实现特定的量程。通过计算所需的串联电阻值,并使用电阻箱进行调节,最终校准改装后的电压表以确保其读数准确。

先将理想电流计(满刻度电流 $I_g$、内阻 $R_g$)与上文分流电阻 $R_1$ 并联,得到量程为 $I_{new}$ 的电流表,其等效内阻
\begin{equation}
    R_A = R_g \parallel R_1 = \frac{R_g R_1}{R_g + R_1}
    = \frac{R_g I_g}{I_{new}}\quad\Big(\text{用 }R_1=\frac{R_g I_g}{I_{new}-I_g}\Big)
\end{equation}
再与串联倍压电阻 $R_3$ 构成电压表。使量程为 $U_{new}$(满刻度时电流为 $I_{new}$),则
\begin{equation}
    R_3 = \frac{U_{new}}{I_{new}} - R_A
        = \frac{U_{new}}{I_{new}} - \frac{R_g I_g}{I_{new}}
        = \frac{U_{new}-R_g I_g}{I_{new}} 
        = \frac{U_{new}}{I_{new}} - \frac{R_g R_1}{R_g + R_1} \, .
\end{equation}
因此我们设计了如\ref{fig:电压表改装电路}的改装电路,将串联电阻 $R_{3}$ 与电压表串联连接。

接下来,我们需要对改装后的电压表进行校准。使用\ref{fig:电压表校准电路}所示电路校验并校准改装电压表,若测量值偏小,则减小电阻箱内阻,反之则增大电阻箱内阻,直至测量值与标准电压表读数相符。记录所得数据并分析误差。

\begin{table}[H]
    \centering
    \begin{minipage}{0.5\textwidth}
    \includegraphics[width=\textwidth]{figures/gaizhuangV.png}
    \subcaption{电压表改装电路示意图}
    \label{fig:电压表改装电路}
    \end{minipage}
    \begin{minipage}{0.4\textwidth}
    \includegraphics[width=\textwidth]{figures/jiaozhunV.png}
    \subcaption{电压表校准电路示意图}
    \label{fig:电压表校准电路}
    \end{minipage}
    \caption{电压表改装与校准示意图}
\end{table}

\subsubsection{设计、改装欧姆表}
本实验要求设计一个改装欧姆表的电路,以实现特定的量程。通过计算所需的串联电阻值,并使用电阻箱进行调节。

设计欧姆表电路如\ref{fig:ohmmeter-circuit}所示。首先短接$a,b$,调节$R_6$使电流计满偏,此时有:
\begin{equation}
I_o = I_g' = \frac{\varepsilon}{R_g'+R'}
\end{equation}

其中$R'$为回路中其他所有电阻之和。
当不同$R_x$接入回路时,有:
\begin{equation}
I_x = \frac{\varepsilon}{R_g'+R'+R_x}
\end{equation}

多次调节$R_x$记录$I_x$和$R_x$的值,并可在电流计面板上刻上刻度以显示不同阻值。这样,我们便得到了一个可以反映被测电阻的欧姆表。特别地,当$I_x=\frac{I_o}{2}$时,有$R_x = R_g' + R'$称为欧姆表的中值电阻。

\begin{figure}[H]
    \centering
    \includegraphics[width=0.6\textwidth]{figures/gaizhuangOhm.png}
    \caption{欧姆表改装示意图}
    \label{fig:ohmmeter-circuit}
\end{figure}

\subsubsection{附加实验}
\begin{enumerate}
    \item 探索利用四种伏安法(分压外接法、分压内接法、限流外接法、限流内接法)来测量电阻,自行搭建电路并进行测量,比较各方法的优缺点及测量结果的准确性。
    \item 探究直流电源的输出功率大小与负载电阻之间的关系,探究负载电阻的功率特性,来验证最大功率传输定理。
\end{enumerate}

\subsection{实验重点(3分)}
\begin{enumerate}
    \item 掌握电流计的工作原理及其内阻的测量方法;
    \item 学会电流表和电压表的设计、改装与校准方法;
    \item 理解欧姆表的工作原理及其改装方法。
\end{enumerate}

\subsection{实验难点(2分)}
\begin{enumerate}
    \item 设计改装电流表和电压表时,计算并调节电阻箱阻值以实现所需量程;
    \item 理解并应用误差分析方法,确保实验结果的准确性和可靠性;
    \item 正确应用电路连接和调试技巧,确保实验过程顺利进行;
    \item 正确进行实验设计和数据分析。
\end{enumerate}
\begin{fullreportonly}
\section{原始数据(20分)}
\begin{figure}[H]
    \centering
    \includegraphics[width=0.8\textwidth]{figures/原始数据.jpg}
    \caption{原始数据}
\end{figure}
\newpage
\section{结果与分析(60分)}
\subsection{数据处理与结果(30分)}

\subsubsection{电流表的改装与校准}
利用内阻为240$\Omega$,满刻度电流为1mA的电流表,设计改装成量程为5mA的电流表。计算所需分流电阻值:
\begin{equation}
    R_{1} = \frac{R_{g} I_{g}}{I_{new} - I_{g}} = \frac{240 \times 1}{5 - 1} = 60 \,\Omega
\end{equation}
将电流表与60$\Omega$的分流电阻并联连接,构成量程为5mA的电流表。使用可调电流源对改装电流表进行校准。记录校准数据如下:
\begin{table}[H]
    % 在这里插入数据处理的内容,如计算过程、数据表格、图表等
    \centering
    \begin{tabular}{|c|c|c|c|c|c|}
        \hline
        $I_{\text{准}}$(mA)&0.98&1.97&2.96&3.94&4.45\\
        \hline
        $I_{\text{改}}$(读数值)(mA)&0.20&0.40&0.60&0.80&0.90\\
        \hline
        $I_{\text{改}}$(理论值)(mA)&1.00&2.00&3.00&4.00&4.50\\
        \hline
        $\Delta I (mA)$&+0.02&+0.03&+0.04&+0.06&+0.05\\
        \hline
        \end{tabular}
    \caption{电流表校准数据处理表}
\end{table}

根据校准时的电流值与理论值的偏差,利用Matlab画出误差折线图,如图所示:
\begin{figure}[H]
    \centering
    \includegraphics[width=0.6\textwidth]{figures/DeltaI_vs_I_theory.png}
    \caption{电流表校准误差折线图}
\end{figure}

\subsubsection{电压表的改装与校准}
利用前一实验得到的5mA的改装电流表,我们将其设计改装成量程为5V的电压表。计算所需串联电阻值:
\begin{equation}
    R_{3} = \frac{U_{new}}{I_{new}} - R_A = \frac{5}{0.005} - \frac{240 \times 1}{5} = 1000 - 48 = 952 \,\Omega
\end{equation}
将改装电流表与952$\Omega$的串联电阻连接,构成量程为5V的电压表。使用可调电压源对改装电压表进行校准。记录校准数据如下:
\begin{table}[H]
    % 在这里插入数据处理的内容,如计算过程、数据表格、图表等
    \centering
    \begin{tabular}{|c|c|c|c|c|c|}
        \hline
        $U_{\text{准}}$(V)&1.05&2.05&3.03&4.01&4.49\\
        \hline
        $I_{\text{改}}$(读数值)(mA)&0.20&0.40&0.60&0.80&0.90\\
        \hline
        $U_{\text{改}}$(理论值)(V)&1.00&2.00&3.00&4.00&4.50\\
        \hline
        $\Delta U (V)$&-0.05&-0.05&-0.03&-0.01&-0.01\\
        \hline
        \end{tabular}
    \caption{电压表校准数据处理表}
\end{table}
根据校准时的电压值与理论值的偏差,利用Matlab画出误差折线图,如图所示:
\begin{figure}[H]
    \centering
    \includegraphics[width=0.6\textwidth]{figures/DeltaU_vs_U_theory.png}
    \caption{电压表校准误差折线图}
\end{figure}

\subsubsection{欧姆表的改装与I-R对应关系测定}
\paragraph{欧姆表组装与I-R曲线测定}
利用实验一的改装电流表,串联一个220$\Omega$的保护电阻、一个滑动变阻器和一个1.5V的干电池,设计改装成欧姆表。
把欧姆表两端接入一个电阻箱$R_x$,先将$R_x$调节为0,调节滑动变阻器使电流计满偏,此时有$I_x=1mA$。
随后调节电阻箱$R_x$,使电流计读数调整至半偏,即$I_x=0.5mA$,此时电阻箱读数为$270\Omega$,说明此时欧姆表整体内阻为$270\Omega$。

随后,调节电阻箱$R_x$的阻值,记录下每个$R_x$对应的电流计读数$I_x$,记录对应数据如下:
\begin{table}[H]
    \centering
    \begin{tabular}{|c|c|c|c|c|c|c|c|}
        \hline
        $R_x(\Omega)$&0&100&200&300&400&500&600\\
        \hline
        $I_x$(mA)&1.00&0.73&0.58&0.47&0.40&0.35&0.31\\
        \hline
        $R_x(\Omega)$&700&1k&2k&3k&5k&10k&30k\\
        \hline
        $I_x$(mA)&0.27&0.21&0.16&0.07&0.04&0.01&0.00\\
        \hline
        \end{tabular}
\end{table}

利用Matlab绘制$I_x$与$R_x$的关系曲线,如下所示:
\begin{table}[H]
    \centering
    \includegraphics[width=0.6\textwidth]{figures/Ix_vs_Rx_linear.png}
    \caption{欧姆表 I–R 关系曲线}\label{Ix_vs_Rx_linear}
\end{table}


\paragraph{模型推导与拟合结果分析}
将电源内阻并入固定串联电阻,记欧姆表的固定内阻为 $R_0=R_g'+R'$,被测电阻为 $R_x$,电源电动势为 $\varepsilon$。改装电流表由 \SI{1}{mA} 表头并联分流电阻得到 \SI{5}{mA} 量程,表头读数 $I_x$ 为总电流的 $1/m$(本实验 $m=5$)。于是
\begin{equation}
    I_x \,=\, \frac{1}{m}\,\frac{\varepsilon}{R_0+R_x} \,=\, \frac{K/m}{R_0+R_x},
\end{equation}
其中 $K\,(\si{\milli\ampere\ohm})$ 为等效比例常数。由此得到两个常用的等价线性化形式:
\begin{equation}
    \frac{1}{I_x} \,=\, \frac{m}{K}\,R_x + \frac{m}{K}\,R_0\,,\qquad
    R_x \,=\, \frac{K/m}{I_x} - R_0\,.
\end{equation}
“半偏”性质仍然成立:当 $R_x=R_0$ 时 $I_x = I_{x0}/2$($I_{x0}$ 为 $R_x=0$ 时表头电流)。

为检验模型,我们在 $m=5$ 前提下对上表数据做了三种回归:
\begin{itemize}
    \item 线性化一:以 $Y=\frac{1}{I_x}$、$X=R_x$ 回归 $Y=(m/K)X+(m/K)R_0$(斜率为 $m/K$,据此可还原 $K$);
    \item 线性化二(按要求):以 $Y=R_x$、$X=\frac{1}{I_x}$ 回归 $Y=(K/m)X- R_0$(斜率对应 $K/m$);
    \item 非线性拟合(推荐):直接在原模型 $\frac{K/m}{R_0+R_x}$ 上做最小二乘。
\end{itemize}

本实验基于上述数据(去除 $10\,\mathrm{k\Omega}$ 与 $30\,\mathrm{k\Omega}$ 两点)得到的数值结果为:
\begin{itemize}
    \item 线性化一($\frac{1}{I_x}$–$R_x$):$R^2=0.972910$,$K=\SI{1075.226541}{\milli\ampere\ohm}$,$R_0=\SI{83.032748}{\ohm}$;
    \item 线性化二($R_x$–$\frac{1}{I_x}$):$R^2=0.972910$,斜率 $K/m=\SI{209.219838}{\milli\ampere\ohm}$,据此 $K=\SI{1046.099191}{\milli\ampere\ohm}$,$R_0=\SI{49.630507}{\ohm}$;
    \item 非线性拟合($I_x=(K/m)/(R_0+R_x)$):$K=\SI{1350.765866}{\milli\ampere\ohm}$,$R_0=\SI{270.145089}{\ohm}$,因而 $K/m=\SI{270.153173}{\milli\ampere\ohm}$。
\end{itemize}

从结果看,两个线性化口径虽然 $R^2$ 接近,但对 $R_0$ 的估计却和真实的“半偏阻值”($270\Omega$)相差甚远,这是因为对 $\frac{1}{I_x}$ 做普通最小二乘会放大小电流处的误差,易产生参数偏差。
因此我们采取直接在原始模型上做非线性回归的方式,能更好地保持物理意义与一致性。且拟合结果也更加出色。

\paragraph{最终关系式}
采用非线性拟合所得参数并考虑 $m=5$,得到 $K=\SI{1350.765866}{\milli\ampere\ohm}$、$R_0=\SI{270.145089}{\ohm}$,因而 $K/m=\SI{270.153173}{\milli\ampere\ohm}$。故
\begin{equation}
    \boxed{\ R_x[\Omega] \approx \frac{K/m}{\,I_x[\mathrm{mA}]\,} - R_0\  \approx \frac{270.153}{\,I_x[\mathrm{mA}]\,} - 270.145\ }.
\end{equation}
若改用 SI 单位($I_x$ 以 A 计),则上述常数相应除以 $1000$,公式同形。

注意到,此时的两个常数与“半偏”性质高度吻合:当 $I_x=0.5\,\mathrm{mA}$ 时,$R_x \approx 270.153/0.5 - 270.145 \approx 540.306 - 270.145 \approx 270.161\,\Omega$,与测量得到的 $270\,\Omega$ 非常接近;
同时,模型中的$K$正是对应了欧姆表的内电源电动势$\varepsilon$,拟合得到$\varepsilon \approx 1.35\,\mathrm{V}$,与标注$1.5V$的干电池的数据也比较吻合且在合理范围内。这充分验证了模型的正确性。

为直观起见,下列图像给出三种回归的可视化对比:
\begin{table}[H]
    \centering
    \begin{minipage}{0.45\textwidth}
            \includegraphics[width=\textwidth]{figures/ohmmeter_linearization.png}
    \subcaption{线性化:$\frac{1}{I_x}$–$R_x$}

    \end{minipage}
    \begin{minipage}{0.45\textwidth}
         \includegraphics[width=\textwidth]{figures/ohmmeter_R_vs_invI_linear.png}
    \subcaption{线性化二(按要求):$R_x$–$\frac{1}{I_x}$ 的线性拟合}

    \end{minipage}
    \caption{欧姆表三种回归方式的图像(上排)}
\end{table}

\begin{figure}[H]
    \centering
        \includegraphics[width=0.55\textwidth]{figures/ohmmeter_I_vs_R_fit.png}
    \caption{$I_x$–$R_x$ 非线性拟合:$I_x=(K/m)/(R_0+R_x)$}

\end{figure}


\subsection{误差分析(20分)}
我们以标准表读数($I_\text{准}$、$U_\text{准}$)作为参考值,令误差定义为 $\Delta I = I_\text{估}-I_\text{准}$、$\Delta U = U_\text{估}-U_\text{准}$。相对误差定义为 $\delta_I=\Delta I/I_\text{准}$、$\delta_U=\Delta U/U_\text{准}$。不确定度方面,采用A类不确定度:
\begin{equation}
    s = \sqrt{\frac{1}{n-1}\sum_{i=1}^n (\Delta_i-\overline{\Delta})^2},\qquad
    u = \frac{s}{\sqrt{n}}\,.
\end{equation}

\paragraph{实验一:电流表的改装与校准}
由表中五组$\Delta I$数据可得:
\begin{align}
    \overline{\Delta I} &= \SI{0.040}{\milli\ampere}, \\
    s_{\Delta I} &\approx \SI{0.0158}{\milli\ampere}, \\
    u(\Delta I) &= \frac{s}{\sqrt{5}} \approx \SI{0.0071}{\milli\ampere}.
\end{align}
用每个点的相对误差 $\delta_I=\Delta I/I_\text{准}$ 的平均值衡量总体相对误差,有:
\begin{equation}
    \overline{\delta_I} = \frac{1}{5}\sum_{i=1}^5 \delta_{I,i} \approx \SI{1.51}{\percent}.
\end{equation}
用 $\overline{I_\text{准}}=\SI{2.86}{\milli\ampere}$ 来估计相对标准不确定度,可得到:
\begin{equation}
    u_\text{rel}(I) = \frac{u(\Delta I)}{\overline{I_\text{准}}} \approx \frac{\SI{0.0071}{\milli\ampere}}{\SI{2.86}{\milli\ampere}} \approx \SI{0.247}{\percent}.
\end{equation}
据此,可给出电流校准误差的结果表达式为:
\begin{equation}
\boxed{\Delta I = \SI{0.040}{\milli\ampere} \pm \SI{0.007}{\milli\ampere}}
\end{equation}
对应平均相对误差约 \SI{1.51}{\percent},相对标准不确定度约 \SI{0.25}{\percent}。

\paragraph{实验二:电压表的改装与校准}
由表中五组$\Delta U$数据可得:
\begin{align}
    \overline{\Delta U} &= \SI{-0.030}{\volt}, \\
    s_{\Delta U} &\approx \SI{0.020}{\volt}, \\
    u(\Delta U) &= \frac{s}{\sqrt{5}} \approx \SI{0.00894}{\volt}.
\end{align}

用每个点的相对误差 $\delta_U=\Delta U/U_\text{准}$ 的平均值衡量总体相对误差,有:
\begin{equation}
    \overline{\delta_U}=\SI{-1.73}{\percent}.
\end{equation}
    以 $\overline{U_\text{准}}=\SI{2.926}{\volt}$ 来估计相对标准不确定度,可得到:
\begin{equation}
    u_\text{rel}(U) = \frac{u(\Delta U)}{\overline{U_\text{准}}} \approx \frac{\SI{0.00894}{\volt}}{\SI{2.926}{\volt}} \approx \SI{0.306}{\percent}.
\end{equation}
据此,可给出电压校准误差的结果表达式为:
\begin{equation}
\boxed{\Delta U = \SI{-0.030}{\volt} \pm \SI{0.009}{\volt}}
\end{equation}
对应平均相对误差约 \SI{-1.73}{\percent},相对标准不确定度约 \SI{0.31}{\percent}。

\paragraph{主要误差来源简述}
在本实验中,主要误差来源包括以下几个方面:
\begin{enumerate}
    \item 仪器精度限制:电流表、电压表和欧姆表本身的精度有限,可能导致读数误差。
    \item 接触电阻:实验过程中连接线和接点的接触电阻可能引入额外的阻值,影响测量精度。
    \item 人为读数误差:在读取仪表读数时,可能存在视差或读数不准确的情况。
\end{enumerate}

\subsection{实验探讨(10分)}
通过本实验,我们深入理解了电流表、电压表和欧姆表的工作原理及其改装方法。在理论层面上,我们学习了如何计算分流电阻和串联电阻以实现所需的测量功能。
在实践层面上,我们通过实际操作改装和校准仪表,掌握了电流表、电压表、欧姆表的搭建步骤,并学会了如何使用可调电源进行校准。通过数据处理和误差分析,我们还学会了如何拟合实验数据,验证理论模型的准确性。

\section{思考题(10分)}
\subsection{为什么不能用万用表欧姆挡测量电源电阻?}
首先,从原理上说,欧姆表是由$\frac{\varepsilon}{I_g}$的若干倍来标称电阻的,当外接电源时,分子就不再是欧姆表内置电源的电动势$\varepsilon$,而是$\varepsilon$与外接电源的电动势之和(差),因此结果是错误的。

其次,欧姆表的满偏电流和内置电源电压都比较小,若外接电源电压较大,或者反接,可能会使欧姆表超量程乃至损坏。
\subsection{为什么不能用欧姆表测量另一表头内阻?}

首先,表头电阻较小,而欧姆表测量精度较低,测量误差比较大;其次,表头允许流过的电流很小,而欧姆表测量时,电流较大,可能会损坏表头。综合考虑,通常不建议用欧姆表测量另一表头内阻。

如果一定要测量的话,可以串联一个电阻箱,调整其阻值使该支路总阻值约为中值电阻,用欧姆表测出总阻值,再减去电阻箱阻值,即为表头内阻。

\subsection{为什么$I_x$与$R_x$为非线性关系?}
    根据前面的推导过程我们可以知道,总电流 $I=\varepsilon/(R_g'+R'+R_x)$,而表头读数 $I_x=I/m$。因此
    $I_x = \frac{1}{m}\,\frac{\varepsilon}{R_g' + R' + R_x} = \frac{K/m}{R_0+R_x}$,
    其中 $K$、$R_0$ 为常数;可见 $I_x$ 与 $R_x$ 呈反比例关系,显然是非线性的。   

    而根据实验数据处理中的拟合结果也验证了这一点,$I_x$—$R_x$ 的关系曲线明显不是一条直线。具体表达式已在数据处理部分给出。
\end{fullreportonly}
\insertnotes
\end{document}
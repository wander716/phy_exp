\documentclass[]{../template/Report}%方括号内写yuxi即生成预习报告
\settemplatedir{../template/}%设置模板路径
% 表格排版增强
\usepackage{tabularx}
\usepackage{booktabs}
\usepackage{multirow}
\usepackage{siunitx}
\sisetup{detect-weight=true,detect-inline-weight=math}

% 图形与浮动体、列表与数学
\usepackage{graphicx}     % 提供 \includegraphics
\usepackage{float}        % 提供 [H] 浮动体位置参数
\usepackage{enumitem}     % 允许 enumerate 使用 [label=...] 等选项
\usepackage{amsmath}      % 提供 \boxed 等数学命令

\exname{用双臂电桥测低电阻} %实验名称
\extable{11} %实验桌号
\instructor{宋斌} %指导教师
\class{信工2401} %班级
\name{姚舜瑜} %姓名
\stuid{3240100532} %学号

\nyear{2025} %年
\nmonth{10} %月
\nday{21} %日
\nweekday{二} %星期几,e.g. \nweekday{三}
\daypart{上午}%上午/下午

\redate{} %如有实验补做,补做日期
\resitu{} %情况说明:

\begin{document}
\maketitle%输出封面

\section{预习报告(10分)}

\subsection{实验综述(5分)}
\subsubsection{实验目的}
\begin{enumerate}
    \item 了解双臂电桥的结构和工作原理。
    \item 学会使用双臂电桥测量低电阻值。
    \item 了解单臂电桥与双臂电桥的关系和区别。
\end{enumerate}
\subsubsection{实验原理}

\begin{enumerate}[label=(\arabic*)]
\item \textbf{实际问题}

在使用惠斯登电桥测量电阻时,连接待测电阻 $R_x$ 和标准电阻 $R_s$ 的导线电阻 $r$ 以及接触电阻会引入额外的电压降,导致测量结果偏高。尤其当 $R_x$ 仅为毫欧级时,导线与接触电阻常与被测值同量级,使相对误差显著放大。
具体而言,两端测量得到的是端到端等效电阻
\begin{equation}
R \approx R_x + r_1 + r_2 
\end{equation}
\begin{figure}[H]
    \centering
    \includegraphics[width=0.4\textwidth]{电阻误差.png}
    \caption{电阻误差}
    \label{fig:wh}
\end{figure}

\item \textbf{解决方案}

双臂电桥采用\textbf{四端子连接}:两个\textbf{电流端子}提供主电流 $I$,两个\textbf{电压端子}直接从电阻体两端取样电压 $V$。由于电压端子所在的回路在电桥平衡时电流极小,因此其自身的接触电阻和引线电阻不会产生显著电压降,其影响可忽略不计。测得
\begin{equation}
R_x = \frac{V_{\text{电压端子}}}{I_{\text{电流端子}}}.
\end{equation}
与两端接入法相比,四端接入法“转移”了附加电阻相对于待测电阻的位置。附加电阻 $r_1$、$r_2$ 已被“转移”到被测电阻之外,而新增的附加电阻 $r_3$、$r_4$ 并不与 $R_x$ 直接串联,因此不会改变待测电阻的阻值。
进而消除了导线电阻和接触电阻对测量结果的影响。
\begin{figure}[H]
    \centering
    \includegraphics[width=0.5\textwidth]{四端子.png}
    \caption{四端子连接示意图}
    \label{fig:four-terminal}
\end{figure}

\item \textbf{电路结构}

双臂电桥的结构如下图所示:
\begin{figure}[H]
    \centering
    \includegraphics[width=0.55\textwidth]{直流双臂电桥.png}
    \caption{双臂电桥测低电阻电路图}
    \label{fig:kelvin}
\end{figure}
如图所示,双臂电桥的关键结构包括:
\begin{itemize}
    \item \textbf{待测低电阻} $R_x$ 和 \textbf{标准低电阻} $R_s$。
    \item \textbf{第一组比率臂}:由电阻 $R_1$ 和 $R_2$ 构成。
    \item \textbf{第二组比率臂}:由电阻 $R_3$ 和 $R_4$ 构成。
    \item \textbf{连接导线电阻} $r$:连接 $R_x$ 和 $R_s$ 之间的导线,其电阻值不可忽略。
    \item \textbf{检流计} $G$:用于判断电桥是否平衡(即 $I_G = 0$)。
    \item \textbf{电源} $\varepsilon$:提供测量电流。
\end{itemize}

\item \textbf{工作原理与公式推导}

双臂电桥的测量核心依然是\textbf{平衡法}。通过调节可变电阻(通常是 $R_1, R_2, R_3, R_4$ 或 $R_s$),使得检流计 $G$ 中没有电流通过($I_G = 0$)。此时,B 点和 D 点的电势相等,即 $V_B = V_D$。

根据电路图,当电桥平衡时:
\begin{enumerate}[label=\roman*.]
    \item B 点的电势可由 $R_1$ 和 $R_2$ 构成的分压电路得到:
    \begin{equation}
    V_B = V_{C_{x1}} - I_1 R_1
    \end{equation}
    其中 $I_1$ 是流过 $R_1$ 和 $R_2$ 的电流。

    \item D 点的电势可由 $R_x$、$R_3$ 和 $R_4$ 构成的回路得到:
    \begin{equation}
    V_D = V_{C_{x1}} - I_3 R_x - I_2 R_3
    \end{equation}
    其中 $I_3$ 是流过 $R_x$ 和 $R_s$ 的主电流,$I_2$ 是流过 $R_3$ 和 $R_4$ 的电流。
\end{enumerate}
由于 $V_B = V_D$,可得:
\begin{equation}
    I_1 R_1 = I_3 R_x + I_2 R_3 \label{eq:kelvin1}
\end{equation}
同理,我们也可以从另一侧($C_{s1}$)分析电势,由于$I_g=0$,电路电流具有显著的对称性,那么就可以得到:
\begin{enumerate}[label=\roman*.]
    \item B 点的电势:$V_B = V_{C_{s1}} + I_1 R_2$
    \item D 点的电势:$V_D = V_{C_{s1}} + I_3 R_s + I_2 R_4$
\end{enumerate}
由于 $V_B = V_D$,可得:
\begin{equation}
    I_1 R_2 = I_3 R_s + I_2 R_4 \label{eq:kelvin2}
\end{equation}
将式 \eqref{eq:kelvin1} 和式 \eqref{eq:kelvin2} 相除,整理得到:
\begin{equation}
    \frac{R_1}{R_2} = \frac{I_3 R_x + I_2 R_3}{I_3 R_s + I_2 R_4} \label{eq:kelvin3}
\end{equation}
在实际的双臂电桥中,我们会刻意让\textbf{两组比率臂的比例相等},即:
\begin{equation}
\frac{R_1}{R_2} = \frac{R_3}{R_4}
\end{equation}
将此条件代入式 \eqref{eq:kelvin3} 并整理,可得:
\begin{equation}
\frac{R_1}{R_2} = \frac{I_3 R_x + I_2 (\frac{R_1}{R_2}R_4)}{I_3 R_s + I_2 R_4}
\end{equation}
\begin{equation}
\frac{R_1}{R_2} (I_3 R_s + I_2 R_4) = I_3 R_x + I_2 \frac{R_1}{R_2} R_4
\end{equation}
消去等式两边的相同项 $\frac{R_1}{R_2} I_2 R_4$,得到:
\begin{equation}
\frac{R_1}{R_2} I_3 R_s = I_3 R_x
\end{equation}
最终得到双臂电桥的平衡条件公式:
\begin{equation}
    \boxed{R_x = R_s \frac{R_1}{R_2}}
\end{equation}

\item \textbf{结论}

从最终的平衡公式可以看出,待测电阻 $R_x$ 的值仅与标准电阻 $R_s$ 以及第一组比率臂的比例 $\frac{R_1}{R_2}$ 有关。
双臂电桥之所以能精确测量低电阻,其核心在于它通过独特的设计:

\begin{enumerate}[label=\roman*.]
    \item \textbf{分离电流与电压路径(四端子法):}
    待测电阻 $R_x$ 和标准电阻 $R_s$ 均采用四端子连接。两个\textbf{电流端子}用于通过主测量电流 $I_3$,而两个\textbf{电压端子}则从电阻核心部分引出电势用于比较。由于电压端子所在的回路在电桥平衡时电流极小,因此其自身的接触电阻和引线电阻不会产生显著电压降,其影响可忽略不计。

    \item \textbf{第二比率臂的补偿作用:}
    即便采用四端子法,连接 $R_x$ 和 $R_s$ 的导线电阻 $r$ 仍然存在于测量回路中,并影响 D 点的电势。双臂电桥引入了第二组比率臂 $R_3, R_4$,它跨接在导线电阻 $r$ 的两端。检流计的一端 D 连接在 $R_3$ 和 $R_4$ 之间,而不是直接连在导线 $r$ 上。

    \item \textbf{数学上的设计:}
    通过设定一个关键的设计条件$\frac{R_1}{R_2} = \frac{R_3}{R_4}$,即让两组比率臂的比例严格相等:
    当此条件满足时,如前的公式推导所示,连接电阻 $r$ 对平衡条件的影响在数学上被完全抵消。最终的平衡公式
    $ R_x = R_s \frac{R_1}{R_2} $
    中不包含 $r$ 项。这意味着,测量结果与导线电阻和接触电阻无关,从而保证了测量的准确性。
\end{enumerate}

\end{enumerate}

\subsubsection{实验内容}
\begin{enumerate}[label=(\arabic*)]
    \item 测量金属导体电阻率
    \begin{enumerate}[label=\roman*.]
        \item 开启双臂电桥与检流计,将灵敏度置最低档,调零检流计。
        \item 按四端子法接入被测导体,选择合适倍率 $M$。先按 B 通电,后按 G 检流,配合阻值粗、细调盘,使指针偏转减小。
        \item 逐步提高检流计灵敏度,在最高档完成平衡并记录读数。计算
        \begin{equation}
            R_x = M \,(R_{\text{粗}} + R_{\text{细}}).
        \end{equation}
        \item 用游标卡尺在多处测量直径 $d$(取平均),用米尺测量长度 $L$。
        \item 根据上述数据,可计算得到电阻率为:
        \begin{equation}
            \rho = \frac{\pi d^2}{4L}R_x .
        \end{equation}
    \end{enumerate}

    \item 测量金属导体电阻温度系数
    \begin{enumerate}[label=\roman*.]
        \item 在室温 $T_0$ 下按最高灵敏度平衡,得到 $R_0$。
        \item 开启恒温装置,设定目标温度,待温度稳定。
        \item 以约 $3\,^\circ\mathrm{C}$ 为步进,记录温度 $T_i$ 与对应电阻 $R_i$ 10 组数据。每次读数前先待温度稳定。
        \item 采用线性拟合,利用Matlab计算得到电阻的温度系数为 $\alpha$。
    \end{enumerate}
\end{enumerate}

\subsection{实验重点(3分)}
\begin{enumerate}
    \item 深刻理解双臂电桥如何通过“四端子法”和引入第二比率臂来消除导线及接触电阻对测量结果的影响。
    \item 熟练掌握电桥的正确操作步骤,能够通过粗调和细调,快速、准确地使电桥达到平衡状态。
    \item 学会根据电桥的平衡条件公式 $R_x = R_s \frac{R_1}{R_2}$,准确计算出待测低电阻的阻值。
\end{enumerate}

\subsection{实验难点(2分)}
\begin{enumerate}
    \item 在电桥接近平衡时,需要适当提高检流计的灵敏度以精确判断。但灵敏度过高或过低都会给观察带来困难,需要准确把握切换时机。
    \item 双臂电桥调节旋钮较多,需要在多个比率档位之间进行尝试。如何高效地进行粗调和细调,快速平衡是本实验的一个难点。
    \item 对于低电阻测量,任何不稳定的接触都可能引入接触电阻,干扰电桥平衡的判断。确保所有接线柱拧紧、接触良好是实验成功的前提。
\end{enumerate}

\begin{fullreportonly}
\section{原始数据(20分)}
\begin{figure}[H]
\centering
\includegraphics[width=0.6\textwidth]{figures/实验数据.jpg}
\caption{实验原始数据记录}
\end{figure}
\section{结果与分析(60分)}
\subsection{数据处理与结果(30分)}
\subsubsection{实验一:测量金属导体电阻率}
根据电阻率的公式$\rho = \frac{\pi d^2}{4L}R_x$,本实验需测量导体的直径 $d$、长度 $L$ 以及电阻 $R_x$。我们使用游标卡尺测出导体的直径,并用电阻丝自带的刻度尺测量其长度。
最后通过双臂电桥测量得到其电阻值 $R_x$。结果如下表:
% 几何尺寸与低电阻一次测量(比率臂0.01)
\begin{table}[H]
    \centering
    \caption{被测导体几何尺寸与低电阻测量(比率臂\,0.01)}
    \label{tab:geom_rx}
    \renewcommand{\arraystretch}{1.2}
    \begin{tabularx}{0.8\linewidth}{>{\centering\arraybackslash}X >{\centering\arraybackslash}X >{\centering\arraybackslash}X}
        \toprule
    直径 $d$/cm & 长度 $L$/cm & 电阻 $R$/\si{0.01\ohm} \\
        \midrule
        0.410 & 30.00 & 0.06660 \\
        \bottomrule
    \end{tabularx}
\end{table}
其中,比率臂 $M=0.01$,代表表中阻值是真实阻值的100倍。因此通过计算可得实际电阻值为
\begin{equation}
R_x = 0.06660 \times 0.01 = 6.66 \times 10^{-4}\,\Omega.
\end{equation}
进而计算得到电阻率为
\begin{equation}
\boxed{\rho = \frac{\pi (0.410\,\mathrm{cm})^2}{4 \times 30.00\,\mathrm{cm}} \times 6.66 \times 10^{-4}\,\Omega = 2.9310 \times 10^{-6}\,\Omega\cdot\mathrm{cm}.}
\end{equation}

\subsubsection{实验二:测量金属导体的电阻温度系数}
本实验旨在测量金属导体的电阻随温度变化的关系,从而求出其温度系数 $\alpha$。 根据电阻温度关系式
\begin{equation}
R_T = R_0 [1 + \alpha t],
\end{equation}
我们在按步长递进的不同温度 $T$ 下测量对应的电阻 $R_T$。通过线性拟合 $R_T$ 对 $T$ 的数据,可以提取温度系数 $\alpha$。 以下是我们测量得到的温度—电阻数据:
% 温度—电阻数据(比率臂0.1)
\begin{table}[H]
    \centering
    \caption{温度—电阻数据(比率臂\,0.1)}
    \label{tab:t_rx}
    \renewcommand{\arraystretch}{1.2}
    \begin{tabularx}{0.8\linewidth}{>{\centering\arraybackslash}p{1.6cm} >{\centering\arraybackslash}X >{\centering\arraybackslash}X}
        \toprule
    序号 & 温度 $T$/\si{\celsius} & 电阻 $R_x$/\si{0.1\ohm} \\
        \midrule
        1 & 35 & 0.04856 \\
        2 & 38 & 0.04895 \\
        3 & 41 & 0.04946 \\
        4 & 44 & 0.05004 \\
        5 & 47 & 0.05060 \\
        6 & 50 & 0.05124 \\
        7 & 53 & 0.05175 \\
        8 & 56 & 0.05231 \\
        9 & 59 & 0.05286 \\
        10 & 62 & 0.05341 \\
        \bottomrule
    \end{tabularx}
\end{table}
其中,比率臂 $M=0.1$,代表表中阻值是真实阻值的10倍。接下来,我们将使用两种方法计算温度系数 $\alpha$。

\begin{enumerate}
    \item \textbf{平均值计算法}
    
    % 可在此处继续进行线性拟合与温度系数求解
在本表数据的基础上,可计算得到温度系数 $\alpha$。
为了避免测量 0\,\si{\celsius} 时的电阻 $R_0$,可以通过做差法消去 $R_0$,得到温度系数 $\alpha$ 的表达式为:
\begin{equation}
\alpha = \frac{R_{t_2} - R_{t_1}}{R_{t_1} t_2 - R_{t_2} t_1}
\end{equation}

根据表格数据计算多组温度系数 $\alpha$ 的值,计算结果如下:
\begin{table}[H]
    \centering
    \caption{相邻两点配对的温度系数计算结果(两点法)}
    \label{tab:alpha_results}
    \renewcommand{\arraystretch}{1.2}
    \begin{tabularx}{0.9\linewidth}{>{\centering\arraybackslash}p{1.9cm} >{\centering\arraybackslash}p{2.4cm} >{\centering\arraybackslash}p{3.4cm} >{\centering\arraybackslash}X}
        \toprule
        配对 $(i,j)$ & $(t_i, t_j)$/\si{\celsius} & $(R_i, R_j)$/\si{\ohm} & $\alpha$/\si{\per\celsius} \\
        \midrule
        (1,6)   & (35, 50) & (0.04856, 0.05124) & $4.223\times10^{-3}$ \\
        (2,7)   & (38, 53) & (0.04895, 0.05175) & $4.462\times10^{-3}$ \\
        (3,8)   & (41, 56) & (0.04946, 0.05231) & $4.560\times10^{-3}$ \\
        (4,9)   & (44, 59) & (0.05004, 0.05286) & $4.501\times10^{-3}$ \\
        (5,10)  & (47, 62) & (0.05060, 0.05341) & $4.484\times10^{-3}$ \\
        \bottomrule
    \end{tabularx}
\end{table}
平均值为:
 $$\boxed{\bar{\alpha}\approx 4.446\times10^{-3}\,\si{\per\celsius}}$$。

    \item \textbf{线性拟合法}
    
    使用 MATLAB 对温度 $T$ 和电阻 $R_x$ 进行线性拟合,得到拟合直线的斜率 $k$ 和截距 $b$。根据线性关系
$R(T) = R_0\,[1+\alpha T]$,拟合结果为
\begin{equation}
R_0 = 4.200\times 10^{-3}\ \si{\ohm}
\end{equation}
$$\boxed{\alpha = 4.379\times 10^{-3}\ \si{\per\celsius}}$$
拟合图像如下图所示:
\begin{figure}[H]
    \centering
    \includegraphics[width=0.6\textwidth]{figures/线性拟合.png}
    \caption{电阻—温度线性拟合图}
    \label{fig:linear_fit}
\end{figure}
\end{enumerate}

\subsection{误差分析(20分)}

\begin{enumerate}
\item\textbf{不确定度公式导出}

测量模型为
\begin{equation}
    \rho = \frac{\pi D^{2}}{4L}\,R ,
\end{equation}
其中 $R$ 为被测电阻、$D$ 为导体直径、$L$ 为长度。先按对数线性化给出推导过程:
\begin{equation}
    g=\ln\rho=\ln\!\left(\tfrac{\pi}{4}\right)+\ln R+2\ln D-\ln L .
\end{equation}
对小量变化作微分,有
\begin{equation}
    \mathrm{d}g = \frac{\mathrm{d}\rho}{\rho} = \frac{\mathrm{d}R}{R} + 2\frac{\mathrm{d}D}{D} - \frac{\mathrm{d}L}{L} .
\end{equation}
若认为 $R,D,L$ 彼此独立,忽略高阶小量,则方差传播为
\begin{equation}
    u^2(g) = \left(\frac{u(R)}{R}\right)^2 + \left(2\,\frac{u(D)}{D}\right)^2 + \left(\frac{u(L)}{L}\right)^2 .
\end{equation}
又因小量近似 $u(g)\approx u(\rho)/\rho$,故
\begin{equation}
    \left( \frac{u(\rho)}{\rho} \right)^2
    = \left(\frac{u(R)}{R}\right)^2
    + \left(2\,\frac{u(D)}{D}\right)^2
    + \left(\frac{u(L)}{L}\right)^2 .
\end{equation}
注意:$L$ 项的负号在平方后消失;只有当输入量相关时,符号才会通过协方差项影响结果。

因此,可得相对不确定度传播公式
\begin{equation}
    \boxed{\left( \frac{u(\rho)}{\rho} \right)^2
    = \left(\frac{u(R)}{R}\right)^2
    + \left(2\,\frac{u(D)}{D}\right)^2
    + \left(\frac{u(L)}{L}\right)^2}
\end{equation}

\item\textbf{各输入量的标准不确定度}

根据实验仪器说明书,估计各输入量的标准不确定度如下:
\begin{equation}
    u(R) = \frac{0.2\%}{\sqrt{3}}\,R_{\text{量程}},\quad
    u(D) = \frac{\SI{0.02}{\milli\meter}}{\sqrt{3}},\quad
    u(L) = \frac{\SI{0.5}{\milli\meter}}{\sqrt{3}}.
\end{equation}
由此我们可以得到各分量$R$、$D$、$L$的不确定度为:
\begin{align*}
    u(R) &\approx 1.734\times10^{-3} \times \SI{6.66e-4}{\ohm} = \SI{1.154e-6}{\ohm},\\
    u(D) &\approx 2.818\times10^{-3} \times \SI{0.410}{\centi\meter} = \SI{1.156e-3}{\centi\meter},\\
    u(L) &\approx 9.622\times10^{-4} \times \SI{30.00}{\centi\meter} = \SI{2.887e-2}{\centi\meter}.
\end{align*}
因此有:
\begin{align*}
    R&= (6.6600 \pm 0.0012)\times10^{-4}\,\si{\ohm},\\
    D&= 0.4100 \pm 0.0012\,\si{\centi\meter},\\
    L&= 30.00 \pm 0.03\,\si{\centi\meter}.
\end{align*}


\item \textbf{数值计算}

实验数据:$D=\SI{0.410}{\centi\meter}$($=\SI{4.10}{\milli\meter}$)、$L=\SI{30.00}{\centi\meter}$($=\SI{300}{\milli\meter}$)、$R_{\text{量程}}=\SI{1.0e-3}{\ohm}$,
则相对分量为:
\begin{align}
    \frac{u(R)}{R} &= \frac{0.2\%\cdot \SI{1.0e-3}{\ohm} /\sqrt{3}}{\SI{6.66e-4}{\ohm}} = 1.734\times10^{-3},\\
    \frac{u(D)}{D} &= \frac{\SI{0.02}{\milli\meter}/\sqrt{3}}{\SI{4.10}{\milli\meter}} = 2.818\times10^{-3},\\
    \frac{u(L)}{L} &= \frac{\SI{0.5}{\milli\meter}/\sqrt{3}}{\SI{300}{\milli\meter}} = 9.622\times10^{-4}.
\end{align}

由此
\begin{equation}
    \frac{u(\rho)}{\rho}
    = \sqrt{(1.734\times10^{-3})^2 + (2\times2.818\times10^{-3})^2 + (9.622\times10^{-4})^2}
    \approx 5.98\times10^{-3}\ (0.598\%).
\end{equation}

\begin{equation}
    u(\rho) \approx 5.98\times10^{-3}\times \SI{2.931e-6}{\ohm\centi\meter}
    = \SI{1.753e-8}{\ohm\centi\meter},\quad
\end{equation}
因此最终得到:
\begin{equation}
    \boxed{\rho = (2.931 \pm 0.018)\times10^{-6}\,\si{\ohm\centi\meter}}
\end{equation}
主要误差来自于直径 $D$ 的测量不确定度,因为其在电阻率公式中被平方放大了影响。
\end{enumerate}



\subsection{实验探讨(10分)}
本次实验,我们通过双臂电桥成功测量了金属导体的电阻率和电阻温度系数。在理论层次中,我们深刻理解了双臂电桥在低电阻测量中的优势,尤其是其通过四端子法和第二比率臂设计,有效消除了导线和接触电阻的影响,确保了测量的准确性。
在实践操作中,我们掌握了电桥的调节技巧,学会了如何通过粗调和细调快速达到平衡状态。此外,我们还学会了如何处理实验数据,包括计算电阻率和温度系数,并进行了误差分析,识别了主要误差来源。

\section{思考题(10分)}
\subsection{双臂电桥与惠斯登电桥有哪些异同}
\noindent\textbf{相同点}
\begin{enumerate}[label=(\arabic*)]
    \item \textbf{测量原理一致}:均为直流电桥的平衡法测量,通过调节比率臂使检流计电流为零($I_G=0$),由比值关系计算未知电阻。
    \item \textbf{误差控制思路相近}:靠匹配比率臂、选择合适电源与检流计灵敏度、接线规范来提高灵敏度与稳定性。
\end{enumerate}

\noindent\textbf{不同点}
\begin{enumerate}[label=(\arabic*)]
    \item \textbf{接线结构}:惠斯登电桥为\textbf{两端子}接法;双臂电桥为\textbf{四端子}接法,将\textbf{电流端}与\textbf{电压端}分离,电压端回路电流近零,因而其接触电阻几乎不产生附加压降。
    \item \textbf{比率臂结构}:双臂电桥在原有比率臂 $R_1,R_2$ 外,增加第二比率臂 $R_3,R_4$,并跨接在引线电阻 $r$ 两端,使检流计接入点位于 $R_3$-$R_4$ 分压节点上,从而实现对 $r$ 的补偿。而惠斯登电桥无此结构。
    \item \textbf{低阻测量能力}:双臂电桥在满足 $R_1/R_2=R_3/R_4$ 时,其平衡式不含 $r$,能有效消除引线/接触电阻对毫欧级低阻的影响;惠斯登电桥则将 $r$ 串入测量,低阻时相对误差显著。
    \item \textbf{适用范围与复杂度}:双臂电桥适合低阻(毫欧级)高精度测量,接线与调节更复杂;惠斯登电桥结构简单,适合中等及较高阻值的普适测量。
\end{enumerate}

\subsection{为什么双臂电桥在测量低电阻时能消除(或减小)附加电阻对测量结果的影响?}
机理来自\textbf{两点}:
\begin{enumerate}[label=(\arabic*)]
    \item \textbf{四端子分离}:电压端子仅采样被测体两端的电位差,且该回路电流近零,故其自身接触/引线电阻的压降可忽略;附加电阻 $r$ 主要位于\textbf{电流通道},不直接叠加到采样电压上。
    \item \textbf{第二比率臂补偿}:参见文中推导式 \eqref{eq:kelvin3}:
    \begin{equation}
        \frac{R_1}{R_2} = \frac{I_3 R_x + I_2 R_3}{I_3 R_s + I_2 R_4} .
    \end{equation}
    令两组比率臂严格匹配 $\tfrac{R_1}{R_2}=\tfrac{R_3}{R_4}$,代入并整理可得
    \begin{equation}
        R_x = R_s\,\frac{R_1}{R_2},
    \end{equation}
    \noindent\textbf{且不含引线电阻 $r$}。这表明补偿支路在数学上抵消了 $r$ 对平衡条件的影响。若匹配存在微小偏差,则产生二阶小量的残余误差,但远小于未补偿情形。
\end{enumerate}

\subsection{如果四端电阻的电流端和电压端接反了,对测量结果会有什么影响?}
\begin{enumerate}[label=(\arabic*)]
    \item \textbf{四端分离失效,读数偏大}:电压采样端转而处在\textbf{高电流路径}上,采样电压包含电流端引线与接触电阻的压降,等效为
    \begin{equation}
        R_{\text{读}} \approx R_x + r_{\text{引线}} + r_{\text{接触}} \,\, (> R_x),
    \end{equation}
    对毫欧级 $R_x$,哪怕数十微欧的引线/接触电阻也会造成显著相对误差。
    \item \textbf{双臂效果受破坏}:在双臂电桥中,电压端接反使检流计节点不再正确反映被测体两端电位分布,第二比率臂对 $r$ 的\textbf{等价补偿关系被破坏},难以达到严格平衡或得到偏差更大的“伪平衡”。
\end{enumerate}
\end{fullreportonly}
\insertnotes
\end{document}
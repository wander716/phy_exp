\documentclass[]{../模板/Report}%方括号内写yuxi即生成预习报告\documentclass[yuxi]{../template/Report}
\settemplatedir{../模板/}%设置模板路径
\sisetup{
    separate-uncertainty = true,
    per-mode = symbol
}
\usepackage{multirow}

\exname{测量液体表面张力系数} %实验名称
\extable{} %实验桌号
\instructor{} %指导教师
\class{} %班级
\name{} %姓名
\stuid{} %学号

\nyear{} %年
\nmonth{} %月
\nday{} %日
\nweekday{} %星期几,e.g. \nweekday{三}
\daypart{}%上午/下午

\redate{} %如有实验补做,补做日期
\resitu{} %情况说明:

\begin{document}
\maketitle%输出封面

\section{预习报告(10分)}
(注:将已经写好的“物理实验预习报告”内容拷贝过来)

\subsection{实验综述(5分)}
(自述实验现象、实验原理和实验方法,包括必要的光路图、电路图、公式等。不超过500字。)
\subsubsection{实验步骤}
1)力敏传感器开机预热
\par
2)清洗圆环、容器
\par
3)容器中加入待测液体(蒸馏水)
\par
4)安装砝码盘,力传感器的定标
\par
5)取下砝码盘,安装铝圆环,测量液体表面张力
\par
6)计算表面张力系数以及不确定度


\subsubsection{实验原理}
\textbf{用拉脱法测定液体表面张力系数}
\begin{figure}[H]
    \centering
    \includegraphics[width=0.6\textwidth]{拉脱法_实验原理示意图.png}
    \caption{拉脱法实验原理示意图}
\end{figure}

\par
将一洁净的圆筒形吊环浸入液体中,然后缓慢地提起吊环,圆筒形吊环将带起一层液膜。使液面收缩的表面张力f沿液面的切线方向,角$\phi$称为湿润角(或接触角)。当继续提起圆筒形吊环时,$\phi$角逐渐变小而接近为零,这时所拉出的液膜的里、外两个表面的张力f均垂直向下,设拉起液膜破裂时的拉力为F,则有:
\par
$F = (m+m_0)g + 2f$ \quad 式中,m为粘附在吊环上的液体的质量,$m_0$为吊环质量。
\par
因表面张力的大小与接触面周边界长度成正比,则有:
\par
$2f = \pi (D_\text{内} + D_\text{外})\times \alpha$ \quad 式中,比例系数$\alpha$称为表面张力系数,单位为N/m,$\alpha$在数值上等于单位长度上的表面张力。
\par
将公式化简得:
$\alpha = \frac{F - (m+m_0)g}{\pi (D_\text{内}+D_\text{外})}$
\par
由于金属膜很薄,被拉起的液膜也很薄,m很小可以忽略,于是公式简化为:
\par
$\alpha = \frac{F - m_0 g}{\pi (D_\text{内}+D_\text{外})}$

\begin{figure}[H]
    \centering
    \includegraphics[width=0.55\textwidth]{拉脱法_实验原理.png}
    \caption{拉脱法实验原理}
\end{figure}

\begin{figure}[H]
    \centering
    \includegraphics[width=0.55\textwidth]{拉脱法_实验原理2.png}
    \caption{拉脱法实验原理}
\end{figure}


\subsection{实验重点(3分)}
(简述本实验的学习重点,不超过100字。)
\par
(1)\textbf{了解液体表面张力的性质}
\par
(2)\textbf{理解液体表面张力的测量原理:}观察拉脱法测定液体表面张力系数的物理过程和物理现象,并用物理学基本概念和定律进行分析和研究,加深对物理规律的认识和理解。
\par
(3)\textbf{掌握微小力的测量}
\par
(4)\textbf{掌握不确定度的分析}


\subsection{实验难点(2分)}
(简述本实验的实现难点,不超过100字。)
\par
(1)\textbf{吊环拉断液柱瞬间读数的准确捕捉:}实验中需在吊环拉断液柱的瞬间读取数字电压表的峰值读数,该时刻极短,若操作或观察不及时,易导致读数偏差。
\par
(2)\textbf{接触角(湿角)对测量的影响:}理想情况下,吊环拉起液膜时湿角$\theta$应接近零,此时表面张力方向垂直向下。但实际中若吊环或液体不洁净,$\theta$角可能不为0,导致张力分解方向偏离理论假设,引入测量误差。
\par
(3)\textbf{仪器洁净度的严格要求:}清洗有机玻璃器皿和吊环”是关键前置步骤。若容器或吊环残留杂质,会改变液体润湿性,影响接触角和液膜稳定性,导致表面张力系数测量不准确。
\par
(4)\textbf{液面升降速度的控制:}需严格控制液面下降速度,并专注观察液膜破裂瞬间的仪表变化。速度过快会导致液膜受力不均或提前破裂,过慢则可能因蒸发等因素改变液体性质。
\par
(5)\textbf{数据处理:}需正确计算表面张力系数及不确定度,确保实验结果的可靠性。

\begin{fullreportonly}
\section{原始数据(20分)}
(将有老师签名的“自备数据记录草稿纸”的扫描或手机拍摄图粘贴在下方,完整保留姓名,学号,教师签字和日期。)
\begin{figure}[H]
    \centering
    \includegraphics[width=0.55\textwidth]{.JPG}
    \caption{原始数据1}
\end{figure}

\begin{figure}[H]
    \centering
    \includegraphics[width=0.55\textwidth]{.JPG}
    \caption{原始数据2}
\end{figure}

\section{结果与分析(60分)}
\subsection{数据处理与结果(30分)}
(列出数据表格、选择适合的数据处理方法、写出测量或计算结果。)
\subsubsection{实验一:用逐差法求仪器的转换系数 K}
初读数$V_0 = 39.0mV$
\begin{table}[H]
    \centering
    \caption{实验一数据记录与处理}
    \begin{tabular}{|p{6em}|c|c|c|c|}
        \hline
        \textbf{砝码个数($500.00 \times 10^{-6}kg$)} & \textbf{增重读数$V'_i$(mV)} & \textbf{减重读数$V''_i$(mV)} & $V_i = \frac{V'_i + V''_i}{2}$(mV) & \textbf{等间距逐差:$\delta V_i = \frac{1}{4}(V_{i+4} - V_i)$(mV)}  \\
        \hline
        0 & 39.0 & 38.2 & 38.6 & \multirow{2}{*}{$\delta V_1=\frac{1}{4}(V_4-V_0) = 6.6$} \\
        \cline{1-4}
        1 & 45.8 & 44.8 & 45.3 &  \\
        \hline
        2 & 51.8 & 51.4 & 51.6 & \multirow{2}{*}{$\delta V_2=\frac{1}{4}(V_5-V_1) = 6.575$} \\
        \cline{1-4}
        3 & 58.8 & 58.0 & 58.4 &  \\
        \hline
        4 & 65.6 & 64.4 & 65.0 & \multirow{2}{*}{$\delta V_3=\frac{1}{4}(V_6-V_2) = 6.6$} \\
        \cline{1-4}
        5 & 72.0 & 71.2 & 71.6 &  \\
        \hline
        6 & 78.2 & 77.8 & 78.0 & \multirow{2}{*}{$\delta V_4=\frac{1}{4}(V_7-V_3) = 6.925$} \\
        \cline{1-4}
        7 & 87.5 & 84.7 & 86.1 &  \\
        \hline
        \multicolumn{5}{|c|}{每500.00mg对应的电子秤的mV读数$\bar{\delta V} = \frac{1}{4}(\delta V_1 + \delta V_2 + \delta V_3 + \delta V_4) = 6.675(mV)$}  \\
        \hline
    \end{tabular}
\end{table}
转换系数 K = $\frac{mg}{\bar{\delta V}} = \frac{500 \times 10^{-6} \times 9.793}{6.675} = 7.336 \times 10^{-4} \quad (N/mV)$

\subsubsection{实验二:用拉脱法求拉力对应的电子秤读数}
水温(室温):20.9℃,电子秤初始读数$V_0 = 42.1(mV)$
\begin{table}[H]
    \centering
    \caption{实验二数据记录与处理}
    \begin{tabular}{|c|c|c|c|}
        \hline
        \textbf{测量次数} & \textbf{拉脱时最大读数$V_1$(mV)} & \textbf{吊环读数$V_2$(mV)} & \textbf{表面张力对应读数$V = V_1 - V_2$(mV)}  \\
        \hline
        1 & 66.2 & 43.0 & 23.2  \\
        \hline
        2 & 67.0 & 43.1 & 23.9  \\
        \hline
        3 & 67.0 & 43.0 & 24.0  \\
        \hline
        4 & 65.0 & 42.6 & 22.4  \\
        \hline
        5 & 65.3 & 42.4 & 22.9  \\
        \hline
        \textbf{平均值} & & & $\bar{V} = 23.28$  \\
        \hline
    \end{tabular}
\end{table}

\subsubsection{实验三:吊环的内、外直径}
\begin{table}[H]
    \centering
    \caption{实验三数据记录与处理}
    \begin{tabular}{|c|c|c|c|c|c|c|}
        \hline
        \textbf{测量次数} & \textbf{1} & \textbf{2} & \textbf{3} & \textbf{4} & \textbf{平均值(mm)}  \\
        \hline
        \textbf{内径$D_\text{内}$(mm)} & 32.80 & 33.00 & 32.84 & 33.02 & $\bar{D_\text{内}}=32.92$  \\
        \hline
        \textbf{外径$D_\text{外}$(mm)} & 34.84 & 34.84 & 34.74 & 34.78 & $\bar{D_\text{外}}=34.80$  \\
        \hline
    \end{tabular}
\end{table}
$\bar{L} = \pi (\bar{D_\text{内}}+\bar{D_\text{外}}) = 212.75mm$

\subsubsection{计算液体表面张力系数}
$\bar{\alpha} = \frac{\bar{K} \times \bar{V}}{\bar{L}} = \frac{7.336\times 10^{-4} \times 23.28}{212.75 \times 10^{-3}}= 8.027 \times10^{-2}N/m$

\subsection{误差分析(20分)}
(运用测量误差、相对误差或不确定度等分析实验结果,写出完整的结果表达式,并分析误差原因。)
\par
\subsubsection{不确定度的计算与分析:}
\par
不确定度计算公式:$\Delta_B(x) = \frac{\Delta_\text{仪}}{p} \quad p = \sqrt{3}$
\par
$\Delta_A(x) = \sqrt{\frac{\sum_{i=1}^{n} (x_i - \bar{x})^2}{n(n - 1)}}$
\quad
$\Delta(x) = \sqrt{\Delta_A(x)^2 + \Delta_B(x)^2}$
\begin{table}[H]
    \centering
    \caption{不确定度计算}
    \begin{tabular}{|c|c|c|c|}
        \hline
         & \textbf{定标时读数$\Delta V/mV$} & \textbf{测力时读数V/mV} & \textbf{接触面长度L/mm}  \\
        \hline
        \textbf{测量仪器} & \textbf{电子秤} & \textbf{电子秤} & \textbf{游标卡尺}  \\
        \hline
        \textbf{量差允许$\Delta_\text{仪}$} & $\pm0.1mV$ & $\pm0.1mV$ & $\pm0.02mm$  \\
        \hline
        \textbf{$\Delta_B(x)$} & $\pm0.1mV$ & $\pm0.1mV$ & $\pm0.01mm$  \\
        \hline
        \textbf{$\Delta_A(x)$} & 0.0835 & 0.3023 & 内径:0.0557;外径:0.0245;$\Delta_A(D) = 0.0609$  \\
        \hline
        \textbf{$\Delta(x)$} & $\Delta(\Delta V)=0.1mV$ & $\Delta(V)=0.3mV$ & $\Delta(L)=\pi \Delta(D) = 0.19mm$  \\
        \hline
        \textbf{测量平均值$\bar{x}$} & $\Delta V = 6.7mV$ & $\bar{V} = 23.3mV$ & $\bar{L}=12.75mm$  \\
        \hline
    \end{tabular}
\end{table}
$\frac{\Delta \alpha}{\bar{\alpha}} = \sqrt{(\frac{\Delta (\Delta V)}{\bar{\Delta V}})^2 + (\frac{\Delta V}{\bar{V}})^2 + (\frac{\Delta L}{\bar{L}})^2} = 0.0198$
\par
结合$\bar{\alpha} = 8.027 \times10^{-2}$,可得$\Delta \alpha = 0.159\times10^{-2}$
\par
液体表面张力系数$\alpha = \bar{\alpha} \pm \Delta \alpha = (8.03 \pm 0.16)\times10^{-2}N/m$

\subsubsection{误差来源分析}
(1)\textbf{仪器误差}
\par
测量工具精度限制:力传感器的灵敏度不足会引入误差。
\par

(2)\textbf{操作误差}
\par
液膜状态影响:拉起液膜时,若速度过快易导致液膜破裂或形成非对称形状。
\par
接触角偏差:实际实验中液体与测量工具(如铂金环)的接触角并非理想的零角,接触角θ的存在会使表面张力计算公式中$\cos \theta$项产生偏差,尤其当液体润湿性较差时误差更显著。
\par

(3)\textbf{环境影响}
\par
温度波动:液体表面张力系数随温度升高而减小,环境温度变化会直接影响测量结果。
\par
液体纯度不足:若液体中含有杂质(如灰尘、油污),会降低表面张力系数,导致测量值偏低;多次测量后液体表面可能残留杂质,需及时更换样品。
\par
外界干扰:实验台振动、气流扰动会导致液面不稳定,或使拉力测量出现波动。

\subsection{实验探讨(10分)}
(对实验内容、现象和过程的小结,不超过100字。)
\par
\par
通过本次实验,我学习了液体表面张力的性质,理解了液体表面张力的测量原理,掌握了微小力的测量方法,学会了利用逐差法处理实验数据,熟练了不确定度的计算。同时,我在实验中学会熟练地使用游标卡尺、电子秤等测量工具。
\par
对于实验原理中不易直接测量的物理量(拉力F),本实验使用硅压阻力敏传感器,实现了将\textbf{微小量放大}。同时,此\textbf{放大}的实现需要我们在正式实验前进行标定。这一实验思路让我深受启发。
\par
该实验锻炼了我的耐心、动手操作能力和数据处理能力,同时对物理学中\textbf{“微小量放大”}的思想有了更深的认识。

\section{思考题(10分)}
(解答教材或讲义或老师布置的思考题,请先写题干,再作答。)
\subsection{在圆环上提水膜即将破裂时$F=mg+F_\text{拉}$成立,若过早读数,对实验结果会有什么影响?}
过早读数会导致测量的表面张力系数偏小。
\par
在拉脱法测量中,圆环上提水膜的过程中,表面张力产生的拉力$F_\text{拉}$会随水膜的拉伸逐渐增大,直至水膜即将破裂时达到最大值(此时$F_\text{拉}=F-mg$,F为弹簧测力计示数,mg为圆环重力)。若过早读数(即水膜未达到即将破裂的临界状态),此时记录的 F 尚未达到最大值,代入公式$\alpha = \frac{F - m_0 g}{\pi (D_\text{内}+D_\text{外})}$,导致表面张力系数$\delta$的测量值比真实值偏小。

\subsection{圆环或玻璃容器不清洁,$\delta$会有什么变化?}
圆环或玻璃容器不清洁会导致测量的液体表面张力系数$\delta$偏小。
\par
杂质引入降低表面张力:圆环或容器表面的油污、灰尘等杂质会溶解或附着在液体表面,破坏液体分子间的内聚力。杂质分子占据液体表面后,减少了液体分子间的相互吸引力,导致液体实际表面张力系数真实值减小。
\par
不清洁的表面会降低液体对圆环的浸润性:使液体与圆环的接触角$\theta$增大(理想情况为$\theta = 0°$,此时$\cos \theta = 1$)。示意图如下:
\begin{figure}[H]
    \centering
    \includegraphics[width=0.55\textwidth]{思考题2分析示意图.JPG}
    \caption{思考题2分析示意图}
\end{figure}

\end{fullreportonly}
\insertnotes
\end{document}
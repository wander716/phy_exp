\documentclass{../template/Report}%方括号内写yuxi即生成预习报告
\settemplatedir{../template/}%设置模板路径

% 等宽表格支持
\usepackage{tabularx}
\usepackage{array}

% 固定长度下划线并可在其上方居中填字:\fillblank[宽度]{内容}
% 例:\textbf{光源距离:}\,\fillblank[1.5cm]{400}\,cm
% 若留空:\fillblank[1.5cm]{}
\newcommand{\fillblank}[2][1.8cm]{%
    \makebox[#1]{\hrulefill}% 画固定长度横线
    \hspace{-#1}\makebox[#1][c]{#2}% 居中覆盖文字
}

\exname{光电效应测定普朗克常数} %实验名称
\extable{} %实验桌号
\instructor{} %指导教师
\class{} %班级
\name{} %姓名
\stuid{} %学号

\nyear{} %年
\nmonth{} %月
\nday{} %日
\nweekday{} %星期几,e.g. \nweekday{三}
\daypart{}%上午/下午

\redate{} %如有实验补做,补做日期
\resitu{} %情况说明:

\begin{document}
\maketitle%输出封面

\section{预习报告(10分)}
(注:将已经写好的“物理实验预习报告”内容拷贝过来)

\subsection{实验综述(5分)}
(自述实验现象、实验原理和实验方法,包括必要的光路图、电路图、公式等。不超过500字。)

\subsubsection{图像说明}
光电效应测定普朗克常量的原理如\cref{guangdian}所示。其中GD是光电管,A为阳极,K为阴极。
G为微电流计,V为电压表,E为电源。根据电路原理,调节滑动变阻器R,可使光电管两端加速电位差$U_{AK}$从$-U$到$+U$连续变化。

实验所用单色光是从低压汞灯光谱中用干涉滤色片过滤得到,
光照射阴极时,根据光电效应,阴极释放电子产生阴极电流,$U_{AK}$越大,阴极电流越大,直到达到某一饱和值$I_H$,称为饱和光电流。$I_H$大小和光强成正比。
\subsubsection{爱因斯坦光电效应方程}
光照射到金属表面时,光子能量 $h\nu$ 被电子吸收。当 $h\nu$ 大于金属的逸出功 $W$ 时,电子会逸出。逸出电子的最大初动能 $E_{k,\max}$ 满足爱因斯坦光电效应方程:
\begin{equation}
    h\nu = W + E_{k,\max} = W + \frac{1}{2}mv^2_{\max} \label{eq:guang}
\end{equation}
这表明光电子的最大动能仅与入射光频率 $\nu$ 有关,而与光强无关,且存在一个截止频率(红限频率) $\nu_0 = W/h$。

\subsubsection{减速电位法与线性关系}
本实验采用“减速电位法”测量光电子的最大初动能,原理如\cref{guangdian}所示。通过在阳极A和阴极K之间施加反向电压 $U_{AK}$,当光电流恰好减小为零时,该反向电压即为遏止电压 $U_a$。此时,反向电场对电子做的功恰好等于电子的最大初动能:
\begin{equation}
    eU_a = E_{k,\max} \quad (U_a \ge 0) \label{eq:chu}
\end{equation}
联立\cref{eq:guang}和\cref{eq:chu}可得 $eU_a = h\nu - W$,整理后得到遏止电压 $U_a$ 与入射光频率 $\nu$ 的线性关系:
\begin{equation}
    U_a(\nu) = \frac{h}{e}\,\nu - \frac{W}{e}
\end{equation}
因此,通过实验测量不同频率 $\nu$ 对应的遏止电压 $U_a$,并对 $U_a$-$\nu$ 数据进行线性拟合,即可从图像的斜率 $\mathrm{k}=h/e$ 求得普朗克常数 $h=e\,\mathrm{k}$,并从截距 $\mathrm{b}=-W/e$ 求得逸出功 $W=-e\,\mathrm{b}$。

\begin{figure}[H]
    \centering
    \includegraphics[width=0.35\textwidth]{guangdian.pdf}
    \caption{实验原理图}
    \label{guangdian}
\end{figure}

\subsection{实验重点(3分)}
(简述本实验的学习重点,不超过100字。)
\begin{enumerate}
    \item \textbf{搞懂核心原理:} 搞清楚爱因斯坦光电效应方程,理解为什么 $U_a$ 和 $\nu$ 成线性关系,这是测量 $h$ 的理论基础。
    \item \textbf{学会仪器操作:} 掌握光电效应仪的使用,特别是“零电流法”测量 $U_a$ 的操作和微电流计的调零。
    \item \textbf{掌握数据拟合:} 学会用Origin 对 $U_a$-$\nu$ 数据进行线性拟合,并能从斜率和截距中正确计算 $h$ 和 $W$。
\end{enumerate}

\subsection{实验难点(2分)}
(简述本实验的实现难点,不超过100字。)
\begin{enumerate}
    \item \textbf{仪器调节:} 首次使用仪器,调整光路(使光斑聚焦于阴极中心)和微电流计调零可能不熟练,容易引入误差。
    \item \textbf{暗电流干扰:} 仪器可能存在本底暗电流或阳极光电流,这会干扰“零电流”点的精确判断,从而影响 $U_a$ 测量的准确性。
\end{enumerate}

\begin{fullreportonly}
\section{原始数据}
% 原始数据如\cref{data}所示。
% \begin{figure}[hbp]
%     \centering
%     \includegraphics[width = 0.8\textwidth]{figures/data.jpg}
%     \caption{实验数据}
%     \label{data}
% \end{figure}


\section{结果与分析}
\subsection{数据处理与结果}
\subsubsection{测量截止电压}

在仪器预热稳定后,我们固定光源距离$L=\SI{400}{mm}$、光阑孔径$\phi = \SI{4}{mm}$,对不同波长的光进行测量。所得截止电压与反向电流数据汇总于\cref{jiezhidianya_grid}。其中,反向电流的测量条件为$U_{AK} = \SI{-2.00}{V}$。

\begin{table}[H]
    \centering
    \caption{不同波长光的截止电压和反向电流}
    \label{jiezhidianya_grid}
    \begin{tabular}{|C{.2\textwidth}|C{.13\textwidth}|C{.13\textwidth}|C{.13\textwidth}|C{.13\textwidth}|C{.13\textwidth}|}
        \hline % 替换 \toprule
        波长$\lambda(\si{nm})$ & 577 & 546 & 436 & 405 & 365 \\
        \hline % 替换 \midrule
        频率$\nu(\times 10^{14}\si{Hz})$ & 5.196 & 5.490 & 6.879 & 7.408 & 8.214 \\
        \hline % <--- 在每行数据后都添加 \hline
        反向电流$I(\times 10^{-13}\si{A})$ & -21 & -32 & -35 & -19 & -37 \\
        \hline % <--- 在每行数据后都添加 \hline
        截止电压$U_a(\si{V})$ & 1.543 & 1.131 & 0.893 & 0.535 & 0.492 \\
        \hline % 替换 \bottomrule
    \end{tabular}
\end{table}

使用Python对截止电压$U_a$与频率$\nu$进行线性拟合,得到拟合直线如\cref{fit1}所示。$R^2 = 0.999315$,说明拟合效果很好。
\begin{figure}[H]
    \centering
    \includegraphics[width=0.6\textwidth]{figures/Ua.png}
    \caption{截止电压与频率的线性拟合}
    \label{fit1}
\end{figure}

其中,拟合的直线斜率$k = \SI{3.8916e-15}{V\cdot s}$,截距$b = \SI{-1.4170}{V}$。
于是得到普朗克常量$\bar{h} = k\cdot e = \SI{6.23e-34}{J\cdot s}$,
逸出功$\bar{W} = -b\cdot e = \SI{1.42}{eV}$,红限频率$\bar{\nu_0} = \frac{\bar{W}}{\bar{h}} = \SI{3.6e14}{Hz}$。


\subsubsection{测量光电管的伏安特性曲线}
两次测定中光电管的伏安数据,根据原始数据作出其伏安特性曲线分别如\cref{fuangraph1}和\cref{fuangraph2}所示。

\begin{figure}[H]
    \centering
    \begin{subfigure}[b]{0.45\textwidth}
        \includegraphics[width=\textwidth]{8_365_380.png}
        \caption{$\phi=\SI{8}{mm},\lambda = \SI{365}{nm},L = \SI{380}{mm}$时的伏安曲线}
        \label{fuangraph1}
    \end{subfigure}
    \hfill
    \begin{subfigure}[b]{0.45\textwidth}
        \includegraphics[width=\textwidth]{4_577_400.png}
        \caption{$\phi=\SI{4}{mm},\lambda = \SI{577}{nm},L = \SI{400}{mm}$时的伏安曲线}
        \label{fuangraph2}
    \end{subfigure}
\caption{光电管伏安特性曲线}
\end{figure}

从图中可以观察到,当阳极电压超过截止电压后,光电流开始出现。
在截止电压附近,光电流随电压增长迅速;
而在电压较高时,增长趋势放缓,接近饱和状态。然而,受限于实验仪器的最大电压,本次测量未能使光电流达到完全饱和。

\subsubsection{探究饱和光电流和光强的关系}
为探究光强与饱和光电流的关系,固定波长$\lambda = \SI{365}{nm}$、以及加上正向电压$U_{AK} = \SI{30.020}{V}$,
分别测量不同距离$L$和不同光阑孔径$\phi$时的饱和光电流,
数据分别如\cref{guangqiangphi}和\cref{guangqiangL}所示。
并分别作出$I_H \sim \phi^2$和$I_H \sim \frac{1}{L^2}$图线,如\cref{guangqiangphigraph}和\cref{guangqiangLgraph}所示。
\begin{table}[H]
    \centering
    \caption{$\lambda = \SI{365}{nm}$、$U_{AK} = \SI{30.02}{V}$、$L = \SI{400}{mm}$时不同光阑孔径$\phi$时的饱和光电流}
    \label{guangqiangphi}
    \begin{tabular}{C{.25\textwidth}C{.12\textwidth}C{.12\textwidth}C{.12\textwidth}}
        \toprule
        光阑孔径$\phi(\si{mm})$ & 2 & 4 & 8 \\
        \midrule
        饱和光电流$I_H(\times 10^{-10}\si{A})$ & 19.1 & 71.3 & 256 \\
        \bottomrule
    \end{tabular}
\end{table}

\begin{table}[H]
    \centering
    \caption{$\lambda = \SI{365}{nm}$、$U_{AK} = \SI{30.02}{V}$、$\phi = \SI{8}{mm}$时不同距离$L$时的饱和光电流}
    \label{guangqiangL}
    \begin{tabular}{C{.25\textwidth}C{.12\textwidth}C{.12\textwidth}C{.12\textwidth}}
        \toprule
        距离$L(\si{mm})$ & 310 & 350 & 400 \\
        \midrule
        饱和光电流$I_H(\times 10^{-10}\si{A})$ & 524 & 370 & 256 \\
        \bottomrule
    \end{tabular}
\end{table}

\begin{figure}[H]
    \centering
    \begin{subfigure}[b]{0.45\textwidth}
        \includegraphics[width=\textwidth]{res2.png}
        \caption{不同光圈值$\phi$和$I_H \sim \phi^2$图线}
        \label{guangqiangphigraph}
    \end{subfigure}
    \hfill
    \begin{subfigure}[b]{0.45\textwidth}
        \includegraphics[width=\textwidth]{guangqiangLgraph.png}
        \caption{不同距离$L$的$I_H \sim \frac{1}{L^2}$图线}
        \label{guangqiangLgraph}
    \end{subfigure}
\caption{饱和光电流与光强的关系}
\end{figure}

拟合结果如下:

\begin{table}[H]
    \centering
    \caption{$I_H$ 与 $\phi^2$ 的线性拟合结果}
    \label{tab:IH_vs_phi2}
    \begin{tabular}{lc}
        \toprule
        拟合参数 & 数值 \\
        \midrule
        斜率 (k) & $3.9196 \times 10^{-10}$ \\
        截距 (b) & 0.0000 \\
        相关系数 (r) & 0.999777 \\
        判定系数 ($R^2$) & 0.999554 \\
        \bottomrule
    \end{tabular}
\end{table}

\begin{table}[H]
    \centering
    % 注意:1/L^2 在 LaTeX 标题中需要用 $...$ 包裹
    \caption{$I_H$ 与 $1/L^2$ 的线性拟合结果}
    \label{tab:IH_vs_L2_inv}
    \begin{tabular}{lc}
        \toprule
        拟合参数 & 数值 \\
        \midrule
        斜率 (k) & $6.4607 \times 10^{-3}$ \\
        截距 (b) & 0.0000 \\
        相关系数 (r) & 0.999190 \\
        判定系数 ($R^2$) & 0.998381 \\
        \bottomrule
    \end{tabular}
\end{table}


从图线和高$R^2$值可以看出,$I_H$与$\phi^2$(正比于光阑面积)及$I_H$与$\frac{1}{L^2}$(正比于光照强度)均表现出强烈的线性关系。此结果验证了饱和光电流与入射光强成正比的理论。

\subsection{误差分析}
\subsubsection{测量截止电压}
% (假设 $k$ 和 $b$ 是由您的 Python 拟合得出的)
% 1. 拟合中心值 (来自 Origin 图像, 并已转换为标准单位)
由Python拟合数据,直线斜率$k = \SI{3.8916e-15}{V\cdot s}$,截距$b = \SI{-1.4170}{V}$。
% 2. 拟合标准偏差 (来自 Origin 图像, 并已转换为标准单位)
拟合的标准偏差为 $s_k = \SI{5.88e-17}{V\cdot s}$,截距$b$的标准偏差$s_b = \SI{0.03961}{V}$。

% (以下是基于您的新数据重新计算的结果)
于是得到普朗克常量$\bar{h} = k\cdot e = \SI{6.23e-34}{J\cdot s}$。
普朗克常量测量结果与公认值$h = \SI{6.62607015e-34}{J\cdot s}$的相对误差是
% 计算: |6.234e-34 - 6.626e-34| / 6.626e-34 = 0.0591...
\[E_h = \frac{|\bar{h}-h|}{h} = 5.91\%\]

普朗克常量的A类不确定度为$u_A(h) = e\cdot s_k = \SI{0.09e-34}{J\cdot s}$。
由文档,B类不确定度为$u_B(h) = \frac{0.03\bar{h}}{\sqrt{3}} = \SI{0.11e-34}{J\cdot s}$。
合成不确定度$u(h) = \sqrt{[u_A(h)]^2 + [u_B(h)]^2} = \SI{0.14e-34}{J\cdot s}$。
% (将 u(h) 修约为一位有效数字: 0.1e-34)
% (将 h_bar=6.23e-34 修约到与不确定度相同的小数位: 6.2e-34)
从而最终测得普朗克常量的结果是:
\[h = (6.2 \pm 0.1)\times 10^{-34} \si{J\cdot s}\]

逸出功$\bar{W} = -b\cdot e = \SI{1.42}{eV}$。
逸出功的A类不确定度为$u_A(W) = e\cdot s_b = \SI{0.04}{eV}$。因整体允差为$3\%$,
B类不确定度为$u_B(W) = \frac{0.03\bar{W}}{\sqrt{3}} = \SI{0.025}{eV}$,合成不确定度$u(W) = \sqrt{[u_A(W)]^2 +[u_B(W)]^2} = \SI{0.05}{eV}$。
因此最终测得金属逸出功的结果是:
\[W = (1.42 \pm 0.05) \si{eV}\]

红限频率$\bar{\nu_0} = \frac{\bar{W}}{\bar{h}} = \SI{3.6e14}{Hz}$。
由传递公式,红限频率的不确定度直接由$h$和$W$的不确定度合成得到
% 计算: 3.64e14 * sqrt((0.0465/1.417)^2 + (0.143/6.234)^2) = 0.15e14
$u(\nu_0) = \bar{\nu_0} \sqrt{\left(\frac{u(W)}{\bar{W}}\right)^2 + \left(\frac{u(h)}{\bar{h}}\right)^2} = \SI{0.1e14}{Hz}$。
% (将 nu_0=3.64e14 修约到与不确定度相同的小数位: 3.6e14)
从而最终测得红限频率是:
\[\nu_0 = (3.6 \pm 0.1) \times 10^{14} \si{Hz}\]

误差的主要来源有:
\begin{enumerate}
    \item \textbf{截止电压的判定偏差:} 实验采用电流在零点附近跳动作为截止判据。此方法依赖主观观察,且微电流计本身存在最小分度限制,导致读数点与真实零电流点可能存在系统偏差。
    \item \textbf{光源稳定性:} 汞灯光源的输出光强若未完全稳定(如预热不足),会导致光电流波动,使得截止电压的测量值在不同时刻存在涨落。
    \item \textbf{统计样本量不足:} 仅使用了5组不同波长的数据点进行线性拟合。较少的数据点使得拟合结果(斜率 $k$ 和截距 $b$)对单点测量误差尤为敏感,增大了A类不确定度。
    \item \textbf{仪器的非理想特性:} 理想光电效应模型未考虑暗电流、阳极反向电流等因素。这些额外电流的存在会干扰截止电压的精确测量。此外,阴极表面可能因氧化等原因导致功函数并非均匀恒定。
\end{enumerate}

\subsubsection{测量光电管的伏安特性曲线}
本部分实验的误差包括:
\begin{enumerate}
    \item \textbf{测量精度限制:} 电流计的量程和精度限制了对微弱光电流(截止电压附近)的准确读数,导致曲线起始部分的形态描绘不准。
    \item \textbf{采样点密度:} 用于绘制I-V曲线的数据点较少,尤其是在曲线斜率变化快的区域,采样不足可能导致曲线形态的细节丢失。
    \item \textbf{光源波动:} 光源强度的任何微小变化都会直接引起光电流的整体浮动,为伏安特性曲线引入噪声。
\end{enumerate}

\subsubsection{探究饱和光电流和光强的关系}
本部分实验的误差包括:
\begin{enumerate}
    \item \textbf{饱和状态判定:} $U_{AK} = \SI{30.020}{V}$ 的电压可能未使光电流达到真正的平台期(完全饱和),导致测得的 $I_H$ 值系统性偏低。
    \item \textbf{几何测量误差:} 距离 $L$ 的测量依赖标尺读数,其分度值较大,引入的读数误差在计算 $1/L^2$ 时会被放大。
    \item \textbf{数据点稀疏:} 仅用3个数据点(针对$\phi$和$L$各3点)进行线性拟合,结论的统计可靠性不强,易受单点随机误差影响。
    \item \textbf{读数稳定性:} 测量 $I_H$ 时电表读数存在跳动,这可能源于光源波动或电路的本底噪声。
\end{enumerate}

\subsection{实验探讨}
通过本次实验,我们不仅系统地验证了光电效应的基本规律和光电管的工作特性,也获得了对光量子理论的直观理解。
在数据处理,特别是线性拟合与不确定度分析方面,加强了对软件使用的掌握,让我们的数据处理能力得到了实践锻炼。
对实验中出现的较大误差进行深入剖析的过程,也提升了我们分析和定位实验误差来源的能力。

\section{思考题}
\subsection{测定普朗克常数的关键是什么?怎样根据光电管的特性曲线选择适宜的测定遏止电压$U_a$的方法?}

测定普朗克常数的核心在于准确获取一系列不同频率 $\nu$ 的入射光所对应的遏止电压 $U_a$。
根据爱因斯坦光电效应方程 $U_a = (h/e)\nu - (W/e)$, $U_a \sim \nu$ 图线的斜率 $k = h/e$,因此精确的斜率是计算 $h$ 的前提。

根据实际测得的伏安特性曲线,选择 $U_a$ 的方法应修正非理想因素:
\begin{enumerate}
    \item \textbf{交点法:} 若光电管存在暗电流,即无光照时仍有微小电流,应分别测量光电流特性曲线和暗电流特性曲线。两条曲线的交点所对应的电压 $U_a'$ 即为修正后的截止电压。
    \item \textbf{拐点法:} 若反向电流较大但在某个电压值 $U_a''$ 处迅速饱和,这通常由阳极材料的光电效应引起。此时应取该拐点 $U_a''$ 作为真实的截止电压。
\end{enumerate}

\subsection{从遏止电压$U_a$与入射光的频率$\nu$的关系曲线中,你能确定阴极材料的逸出功吗?}
能够确定。根据光电效应方程 $U_a = (h/e)\nu - (W/e)$,该线性关系的纵轴截距$b = -W/e$。因此,一旦通过拟合得到截距 $b$ 的值,便可通过 $W = -be$ 计算出阴极材料的逸出功。

\subsection{本次实验存在哪些误差来源?实验中如何解决这些问题?}
\begin{enumerate}
    \item \textbf{遏止电压的精确判定:} 针对电流计精度不足,可采用多次测量取平均值的方法来减小随机误差。若条件允许,应换用更高精度的微电流计。
    \item \textbf{拟合数据点不足:} 仅有5个数据点导致拟合的统计涨落较大。应增加不同波长的滤光片,获取更多数据点,以提高拟合参数的准确性。
    \item \textbf{光电管的非理想特性:} 针对暗电流或反向电流的影响,不应简单以“电流为零”为判据,而应采用更严谨的“交点法”或“拐点法”来确定真实的遏止电压。
    \item \textbf{读数波动:} 为确保光源稳定,必须保证汞灯和测量仪器有足够的预热时间。测量环境应避免杂散光干扰。
    \item \textbf{曲线描绘不精:} 在测量伏安特性曲线或光强关系时,应增加采样点密度,特别是在曲线斜率变化快的区域,以便更精确地描绘曲线形态。
    \item \textbf{几何测量误差:} 测量距离 $L$ 时,应尽量使用最小分度值更小的标尺,并确保读数时视线垂直,以减小视差。
\end{enumerate}
\end{fullreportonly}
\insertnotes
\end{document}
\documentclass[]{../template/Report}%方括号内写yuxi即生成预习报告\documentclass[yuxi]{../template/Report}
\settemplatedir{../template/}%设置模板路径
\usepackage{diagbox}
\exname{光的偏振应用研究} %实验名称
\extable{} %实验桌号
\instructor{} %指导教师
\class{} %班级
\name{} %姓名
\stuid{} %学号

\nyear{} %年
\nmonth{} %月
\nday{} %日
\nweekday{} %星期几,e.g. \nweekday{三}
\daypart{}%上午/下午

\redate{} %如有实验补做,补做日期
\resitu{} %情况说明:
\begin{document}
\maketitle%输出封面
\section{预习报告(10分)}
(注:将已经写好的“物理实验预习报告”内容拷贝过来)

\subsection{实验综述(5分)}
(自述实验现象、实验原理和实验方法,包括必要的光路图、电路图、公式等。不超过500字。)
\subsubsection{实验原理}
光波是电磁横波,其传播方向由电场矢量 $\vec{E}$ 与磁场矢量 $\vec{H}$ 的叉积 $\vec{E} \times \vec{H}$ 确定。在光学中,电场矢量 $\vec{E}$ 被称为光矢量,光强 $I$ 与光矢量振幅的平方 $E^2$ 成正比。

根据电磁场理论的边值条件,可以推导出光波在两种各向同性介质分界面上反射时,反射波电场分量与入射波电场分量之比,即菲涅耳公式:

\begin{figure}[H]
    \centering
    \begin{subfigure}[b]{0.45\textwidth}
        \includegraphics[width=\textwidth]{垂直.pdf}
        \caption{$\vec{E}$垂直于入射面的分量}
        \label{pic:ve}
    \end{subfigure}
    \hfill
    \begin{subfigure}[b]{0.45\textwidth}
        \includegraphics[width=\textwidth]{平行.pdf}
        \caption{$\vec{E}$平行于入射面的分量}
        \label{pic:pa}
    \end{subfigure}
\caption{光在介面反射光和入射光的光强之比}
\end{figure}

\begin{enumerate}
    \item \textbf{$\vec{E}$ 垂直于入射面 (s-分量)}

    当电场矢量 $\vec{E}$ 垂直于入射面时(s-分量,如\cref{pic:ve}),反射系数为:
    \begin{equation}
        \frac{E_1}{E_0} = - \frac{\sin(\theta_0 - \theta_2)}{\sin(\theta_0 + \theta_2)}\label{eq:1}
    \end{equation}

    \item \textbf{$\vec{E}$ 平行于入射面 (p-分量)}

    当电场矢量 $\vec{E}$ 平行于入射面时(p-分量,如\cref{pic:pa}),反射系数为:
    \begin{equation}
    \frac{E_1}{E_0} = \frac{\tan(\theta_0 - \theta_2)}{\tan(\theta_0 + \theta_2)}\label{eq:2}
    \end{equation}
\end{enumerate}

由\cref{eq:2}可知,在 $\theta_0 + \theta_2 = \ang{90}$ 的特定条件下(此时 $\tan(\theta_0 + \theta_2) \to \infty$),平行于入射面的电场分量 $E_1$ 为零,即该分量不发生反射。
此时,反射光仅包含垂直于入射面的电场分量,成为线偏振光。满足此条件的入射角 $\theta_0$ 称为布儒斯特角。

偏振片是一种光学元件,它利用二向色性(选择性吸收)原理,将入射的非偏振光转换为线偏振光。其内部存在一个特定方向,称为偏振化方向,只有振动方向平行于该轴的光矢量分量能够通过,而垂直于该轴的分量则被强烈吸收。

当一束强度为 $I_0$ 的线偏振光垂直入射到偏振片上时,若其光矢量振动方向与偏振片的偏振化方向成 $\alpha$ 角,则透射光的光强 $I$ 遵循马吕斯定律:
\begin{equation}
I = I_0 \cos^2 \alpha
\end{equation}
其中 $\alpha$ 为入射光光矢量方向与偏振片偏振化方向的夹角。
\subsubsection{实验内容}
\begin{figure}[H]
    \centering
    \includegraphics[width=0.6\textwidth]{实验装置.pdf}
    \caption{实验装置示意图}
    \label{equip}
\end{figure}
    \paragraph{测量黑色平板的布儒斯特角 ($\alpha_0$)}

    如\cref{equip}所示,调节光路准直。调节入射光使其为 p-偏振光(即偏振方向平行于入射面)。旋转旋臂,使 p-偏振光以不同角度入射,并调节光电池位置以接收反射光。
    观测万用表示数(光电流),寻找反射光强最小的位置。

    在平台角度盘的左、右两侧分别读取实现光强最小的时候旋臂的角度 $\theta_1$ 和 $\theta_2$。布儒斯特角可由 $\alpha_1 = \frac{1}{2} |\theta_1 - 180^\circ|$ 与 $\alpha_2 = \frac{1}{2} |\theta_2 - 180^\circ|$ 确定。取平均值,即:
    \begin{equation}
        \alpha_0 = \frac{1}{2} |\theta_2 - \theta_1|
    \end{equation}
    则该黑色平板的折射率 $n$ 可由布儒斯特定律计算得出:
    \begin{equation}
        n = \tan \alpha_0
        \label{eq:6}
    \end{equation}
    \paragraph{确定偏振片的偏振化方向}\label{1}
    在\cref{equip}中反射光射入的位置放置一偏振片。保持入射角为上一步测得的布儒斯特角 $\alpha_0$。根据菲涅耳公式,此时 p-偏振光被完全透射,反射光为纯粹的 s-偏振光。

    旋转检偏器 P,观测光电流信号。当光电流达到最大值时,检偏器的偏振化方向与 s-偏振光的振动方向平行。记录此时检偏器指针的读数 $\varphi_0$。
    \paragraph{验证光电流和入射光强的关系}
    在之前的基础上,以 $\varphi_0$ 为基准,旋转偏振片,使其透振轴与 s-偏振光方向(即最大光强方向)的夹角为 $\varphi$。

    测量并记录一系列不同 $\varphi$ 角度所对应的光电流 $i(\varphi)$。绘制 $i - \cos^2 \varphi$ 关系图,分析光电流 $i$ 与 $\cos^2 \varphi$ 之间的线性关系,以验证马吕斯定律。
\subsection{实验重点(3分)}
(简述本实验的学习重点,不超过100字。)
\begin{enumerate}
    \item 设计一种可以测量不透明物体的折射率的方法
    \item 研究光电流和入射光强的关系
    \item 线偏振光入射黑色平板时,设计实验测量反射率
\end{enumerate}
\subsection{实验难点(2分)}
(简述本实验的实现难点,不超过100字。)
\begin{enumerate}
    \item 环境光强会变化,测量出来的角度和光电流可能会有偏差
    \item 偏振片转动幅度小时,由于仪器精度的问题,可能读数不会发生改变
\end{enumerate}
\begin{fullreportonly}
\section{原始数据(20分)}
(将有老师签名的“自备数据记录草稿纸”的扫描或手机拍摄图粘贴在下方,完整保留姓名,学号,教师签字和日期。)

\section{结果与分析(60分)}
\subsection{数据处理与结果(30分)}
(列出数据表格、选择适合的数据处理方法、写出测量或计算结果。)
\subsubsection{测量黑色平板折射率}
\begin{table}[H]
\centering
\caption{测量黑色平板折射率}
\label{tab:n}
\begin{tabular}{|l|l|l|l|}
\hline
组数 & \begin{tabular}[c]{@{}l@{}}左侧角坐标\\ $\theta_1$\end{tabular} & \begin{tabular}[c]{@{}l@{}}右侧角坐标\\ $\theta_2$\end{tabular} & 布儒斯特角 \\ \hline
1 & \ang{76.1} & \ang{295.0} & \ang{54.7} \\ \hline
2 & \ang{76.0}   & \ang{291.0} & \ang{53.8}  \\ \hline
3 & \ang{75.0}   & \ang{296.0} & \ang{55.2}  \\ \hline
4 & \ang{75.9} & \ang{295.0} & \ang{54.8} \\ \hline
5 & \ang{76.0}   & \ang{297.0} & \ang{55.2}  \\ \hline
6 & \ang{75.2} & \ang{296.0} & \ang{55.2}  \\ \hline
\end{tabular}%
\end{table}
由\cref{tab:n},布儒斯特角的平均值为
\[\overline{\alpha_0} = \frac{\alpha_1 + \alpha_2 + \cdots + \alpha_6}{6} = \ang{54.8}\]
其中A类不确定度为
\[
u_A = \sqrt{\frac{1}{n(n - 1)} \sum_{i=1}^n (\alpha_i - \overline{\alpha_0})^2} = \ang{0.55}
\]

B类不确定度取仪器最小刻度,即
\[u_B = \frac{\ang{1}}{\sqrt{3}} = \ang{0.6}\]
则布儒斯特角的综合不确定度为
\[u_C = \sqrt{u_A^2 + u_B^2} = \ang{0.8}\]
因此,布儒斯特角的测量结果为
\[\alpha_0 = \overline{\alpha_0} \pm u_C = \ang{54.8} \pm \ang{0.8}\]
由布儒斯特定律(\cref{eq:6}),黑色平板的折射率为:
\[
n = \tan \alpha = 1.42 \pm 0.04
\]
\
\subsubsection{偏振片的偏振化方向}
\begin{table}[H]
\centering
\caption{测量偏振片偏振化角度}
\label{tab:ang}
\begin{tabular}{|l|l|l|l|l|}
\hline
\diagbox{组数}{角度}{光电流} & 极大值  & 极小值   & 极大值   & 极小值   \\ \hline
1  & \ang{72.0} & \ang{159.0} & \ang{259.0} & \ang{338.0} \\ \hline
2  & \ang{71.5} & \ang{159.5} & \ang{257.0} & \ang{340.0} \\ \hline
\end{tabular}
\end{table}
将每个角度持续减去\ang{90},则得到
\begin{table}[H]
\centering
\caption{测量偏振片偏振化角度}
\label{tab:ang:final}
\begin{tabular}{|l|l|l|l|l|}
\hline
\diagbox{组数}{角度}{光电流} & 极大值  & 极小值   & 极大值   & 极小值   \\ \hline
1  & \ang{72.0} & \ang{69.0} & \ang{79.0} & \ang{68.0} \\ \hline
2  & \ang{71.5} & \ang{69.5} & \ang{77.0} & \ang{70.0} \\ \hline
\end{tabular}
\end{table}
根据该数据,计算平均值,A类不确定度,B类不确定度以及合成不确定度:
\[\overline{\beta} = \frac{\beta_1 + \beta_2 + \cdots + \beta_6}{6} = \ang{72.0}\]
A类不确定度为角度的标准偏差:
\[u_A = \sqrt{\frac{1}{n(n - 1)} \sum_{i=1}^n (\beta_i - \overline{\beta})^2} = \ang{1.4}\]
B类不确定度取仪器的最小分度值\ang{;1;}
\[u_B = \frac{\ang{1}}{\sqrt{3}} = \ang{0.6}\]
则合成不确定度为:
\[u = \sqrt{u_A^2 + u_B^2} = \ang{1.5}\]
则得到最终结果:
\[\beta = \ang{72.0} \pm \ang{1.5}\]
\subsubsection{判断光电流和光强的关系}
\begin{table}[H]
\centering
\caption{光电流和入射光强的关系}
\label{tab:intense}
\begin{tabular}{|l|l|l|}
\hline
组数 & 角度  & 光电流大小 \\ \hline
1  & \ang{22}  & \SI{116.4}{\micro \ampere} \\ \hline
2  & \ang{32}  & \SI{157.6}{\micro \ampere} \\ \hline
3  & \ang{42}  & \SI{197.0}{\micro \ampere} \\ \hline
4  & \ang{52}  & \SI{231.0}{\micro \ampere} \\ \hline
5  & \ang{62}  & \SI{249.0}{\micro \ampere} \\ \hline
6  & \ang{72}  & \SI{255.0}{\micro \ampere} \\ \hline
7  & \ang{82}  & \SI{246.0}{\micro \ampere} \\ \hline
8  & \ang{92}  & \SI{221.0}{\micro \ampere} \\ \hline
9  & \ang{102} & \SI{185.0}{\micro \ampere} \\ \hline
10 & \ang{112} & \SI{142.9}{\micro \ampere} \\ \hline
\end{tabular}
\end{table}

分别计算\[\Delta \varphi = \varphi - \beta\]可以画出
\begin{figure}
    \centering
    \includegraphics[width=.8\textwidth]{Code_Generated_Image.png}
    \caption{光电流和光强的关系}
\end{figure}
从图中可以看见,光电流大小和光强近似成线性关系。
\subsection{误差分析(20分)}
(运用测量误差、相对误差或不确定度等分析实验结果,写出完整的结果表达式,并分析误差原因。)

误差可能出现的原因:
\begin{enumerate}[leftmargin = 6em]
    \item 背景的光强不为0,测量得到的光电流可能会有偏移的情况

    该误差可以通过关闭实验光,测量环境光,并在之后实验时将环境光强减去。
    \item 探测器旋臂有宽度,刀口并不是正对刻度,导致读数角度并不是实际的角度

    该误差可以类似分光计避免偏心误差的方式,左右两边同时测量一次取平均值,消除该误差。
    \item 指针在一定角度偏移时,光电流精度较小,示数不变,无法确定极值位置

    该误差可以通过测量两次光电流不变的位置,取平均值记下,来尽量减小。
\end{enumerate}
\subsection{实验探讨(10分)}
(对实验内容、现象和过程的小结,不超过100字。)
本次实验收获颇丰。实验设计巧妙地结合了两个核心物理原理。

在测定折射率的实验中,我们利用了光的偏振特性,通过观测反射光强度随角度的变化来确定布儒斯特角,并据此成功计算出了黑色平板的折射率。

在另一部分实验中,我们使用了硅光二极管作为传感器,实现了将光信号转化为电信号的测量,并通过数据验证了光电流与光照强度之间的对应关系。

此次实验使我对光的波动,光的偏振等知识有了更深入、更融会贯通的理解。
\section{思考题(10分)}
(解答教材或讲义或老师布置的思考题,请先写题干,再作答。)
\subsection{请简述减小实验误差的方法}
针对背景光强的问题,可以通过关闭实验光,测量环境光,并在之后实验时将环境光强减去。

针对无法准确读取旋臂角度的问题,可以采用类似分光计避免偏心误差的方式,左右两边同时测量一次取平均值,消除该误差。

针对光电流精度较小,无法准确度数的问题,可以通过测量两次光电流不变的位置,取平均值记下,来尽量减小。
\subsection{在探究光电流和光照强度关系的时候,背景光照的影响是怎样的}
背景光强可能导致无实验光照射时,光电流传感器仍然有微弱的读数,
导致测量光电流和光强关系时,直线不过原点,而是有一定偏差。

可以在无实验光照射时先读出光电流强度,在之后实验时减去该读数,使得光电流的零点不漂移。

\subsection{选择题}
下列哪些波能够发生偏振现象?
\begin{table}[H]
\centering
\begin{tabular}{C{.2\textwidth}C{.2\textwidth}C{.2\textwidth}C{.2\textwidth}}
A. 声波 & B. 电磁波 & C. 地震波P波 & D. 水波
\end{tabular}
\end{table}
电磁波可以发生偏振。
\begin{enumerate}[label=\Alph*. , leftmargin = 6em]
    \item 声波为纵波,它的振动方向和波传播的方向平行,没办法产生偏振现象。
    \item 电磁波是横波,其中电场和磁场相互垂直振动,且垂直于传播方向,所以可以发生偏振现象。
    \item 地震P波是纵波,可以在固体,液体和气体中传播,但是振动类似于弹簧挤压,振动方向平行于传播方向,所以不能产生偏振现象。

    *但是S波是横波,只能在固体中传播,可以发生偏振现象。
    \item 水波是横波和纵波的合成波,虽然其振动有垂直于传播方向的分量,但其偏振现象不像电磁波那样明显。
\end{enumerate}
\subsection{如何调整两块平行放置的偏振片的透振方向,让通过的太阳光的光强获得最大值和最小值}
当非偏振的太阳光通过第一块偏振片后,光强变为 $I_1 = \frac{1}{2} I_0$,且变为线偏振光。
这束线偏振光再通过第二块偏振片,设两块偏振片的透振方向夹角为 $\theta$,
根据马吕斯定律,
最终透射光强为:
$$I = I_1 \cos^2(\theta) = \frac{1}{2} I_0 \cos^2(\theta)$$

\textbf{获得最大值:}要使光强 $I$ 最大,需要 $\cos^2(\theta)$ 最大,即 $\cos^2(\theta) = 1$。

此时 $\theta = \ang{0}$ 或 \ang{180}。

操作:调整两块偏振片,使它们的透振方向相互平行。

\textbf{获得最小值:}要使光强 $I$ 最小,需要 $\cos^2(\theta)$ 最小,即 $\cos^2(\theta) = 0$。
此时 $\theta = \ang{90}$ 或 \ang{270}。

操作:调整两块偏振片,使它们的透振方向相互垂直。

此时,理想情况下透射光强为0。
\end{fullreportonly}
\insertnotes
\end{document}